% Part: sets-functions-relations
% Chapter: functions
% Section: kinds

\documentclass[../../../include/open-logic-section]{subfiles}

\begin{document}

\olfileid{sfr}{fun}{kin}
\olsection{Tipos de Funciones}

\begin{explain}
Será útil introducir una especie de taxonomía para algunos de los
tipos de funciones que encontramos con mayor frecuencia.

Para empezar, podríamos considerar funciones que tienen la propiedad
de que cada miembro del codominio es un valor de la función. Tales
funciones se denominan !!{sobreyectivas}, y se pueden representar como en
\olref{fig:surjective}.

\begin{figure}
  \olasset{assets/diagrams/surjective.tikz}
  \caption{!!^a{Una función sobreyectiva} tiene cada !!{elemento} del
    codominio como un valor.}
  \ollabel{fig:surjective}
\end{figure}
\end{explain}

\begin{defn}[Función !!^{sobreyectiva}]
Una función $f \colon A \rightarrow B$ es \emph{!!{sobreyectiva}} si y solo si $B$
es también el rango de~$f$, es decir, para cada $y \in B$ existe al menos
un $x \in A$ tal que~$f(x) = y$, o en símbolos:
\[
  (\forall y \in B)(\exists x \in A)f(x) = y.
\]
Llamamos a tal función !!una{sobreyección} de $A$ a $B$.
\end{defn}

\begin{explain}
Si quieres demostrar que $f$ es !!una{sobreyección}, entonces necesitas demostrar
que cada objeto en el codominio de $f$ es el valor de $f(x)$ para alguna
entrada $x$.

Ten en cuenta que cualquier función \emph{induce} !!una{sobreyección}. Después de todo,
dada una función $f \colon A \to B$, sea $f' \colon A \to \ran{f}$ definida por
$f'(x) = f(x)$. Dado que $\ran{f}$ está \emph{definido} como
$\Setabs{f(x) \in B}{x \in A}$, esta función $f'$ está garantizada de ser
!!una{sobreyección}.
\end{explain}

\begin{explain}
Ahora, cualquier función mapea cada entrada posible a una salida única. Pero
también hay funciones que nunca mapean entradas diferentes a las mismas
salidas. Tales funciones se denominan !!{inyectivas}, y se pueden representar
como en \olref{fig:injective}.
\begin{figure}
  \olasset{assets/diagrams/injective.tikz}
  \caption{!!^a{Una función inyectiva} nunca mapea dos argumentos diferentes
    al mismo valor.}
  \ollabel{fig:injective}
\end{figure}
\end{explain}

\begin{defn}[Función !!^{inyectiva}]
Una función $f \colon A \rightarrow B$ es \emph{!!{inyectiva}} si y solo si para
cada $y \in B$ existe como máximo un $x \in A$ tal que~$f(x) = y$. Llamamos
a tal función !!una{inyección} de $A$ a~$B$.
\end{defn}

\begin{explain}
Si quieres demostrar que $f$ es !!una{inyección}, necesitas demostrar que
para cualquier !!{elemento} $x$ e $y$ del dominio de $f$, si $f(x)=f(y)$, entonces
$x=y$.
\end{explain}

\begin{ex}
La función constante $f\colon \Nat \to \Nat$ dada por $f(x) = 1$ no es
!!{inyectiva} ni !!{sobreyectiva}.

La función identidad $f\colon \Nat \to \Nat$ dada por $f(x) = x$ es tanto
!!{inyectiva} como !!{sobreyectiva}.

La función sucesor $f \colon \Nat \to \Nat$ dada por $f(x) = x+1$ es
!!{inyectiva} pero no !!{sobreyectiva}.

La función $f \colon \Nat \to \Nat$ definida por:
\[
  f(x) =
  \begin{cases}
    \frac{x}{2} & \text{si $x$ es par} \\
    \frac{x+1}{2} & \text{si $x$ es impar.}
  \end{cases}
\]
es !!{sobreyectiva}, pero no !!{inyectiva}.
\end{ex}

\begin{explain}
A menudo, queremos considerar funciones que son tanto
!!{inyectivas} como !!{sobreyectivas}. Llamamos a tales funciones
!!{biyectivas}. Se ven como la función representada en
\olref{fig:bijective}. Las !!^{biyecciones} también se denominan a veces
\emph{correspondencias uno a uno}, ya que emparejan de forma única los elementos
del codominio con los elementos del dominio.
\begin{figure}
  \olasset{assets/diagrams/bijective.tikz}
  \caption{!!^a{Una función biyectiva} empareja de forma única los elementos del
    codominio con los del dominio.}
  \ollabel{fig:bijective}
\end{figure}
\end{explain}

\begin{defn}[!!^{biyección}]
Una función $f \colon A \to B$ es \emph{!!{biyectiva}} si y solo si es tanto
!!{sobreyectiva} como !!{inyectiva}. Llamamos a tal función
!!una{biyección} de $A$ a~$B$ (o entre $A$ y~$B$).
\end{defn}

\end{document}
