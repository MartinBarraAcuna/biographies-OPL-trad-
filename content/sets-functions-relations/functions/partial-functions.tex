% Part: sets-functions-relations
% Chapter: functions
% Section: partial-functions

\documentclass[../../../include/open-logic-section]{subfiles}


\begin{document}

\olfileid{sfr}{fun}{par}

\olsection{Funciones Parciales}

\begin{explain}
A veces es útil relajar la definición de función para que no se
requiera que la salida de la función esté definida para todas las
entradas posibles. Tales mapeos se denominan \emph{funciones parciales}.
\end{explain}

\begin{defn}
Una \emph{función parcial} $f \colon A \pto B$ es un mapeo que
asigna a cada !!{elemento} de~$A$ como máximo un !!{elemento} de~$B$.
Si $f$ asigna un elemento de~$B$ a $x \in A$, decimos que $f(x)$ está
\emph{definido}, y de lo contrario \emph{indefinido}. Si $f(x)$ está definido,
escribimos $f(x) \fdefined$, de lo contrario $f(x) \fundefined$. El
\emph{dominio} de una función parcial~$f$ es el subconjunto de~$A$ donde está
definido, es decir, $\dom{f} = \Setabs{x \in A}{f(x) \fdefined}$.
\end{defn}

\begin{ex}
Cada función $f\colon A \to B$ es también una función parcial. Las
funciones parciales que están definidas en todas partes en~$A$---es decir, lo que
hasta ahora hemos llamado simplemente una función---también se denominan funciones \emph{totales}.
\end{ex}

\begin{ex}
La función parcial $f \colon \Real \pto \Real$ dada por $f(x) = 1/x$
está indefinida para $x = 0$, y definida en todas partes.
\end{ex}

\begin{prob}
Dado $f\colon A \pto B$, define la función parcial $g\colon B \pto
A$ por: para cualquier $y \in B$, si existe un único $x \in A$ tal que
$f(x) = y$, entonces $g(y) = x$; de lo contrario $g(y) \fundefined$. Demuestra que si $f$ es inyectiva, entonces $g(f(x)) = x$ para todo $x \in \dom{f}$, y $f(g(y)) = y$ para todo $y \in \ran{f}$.
\end{prob}

\begin{defn}[Gráfica de una función parcial]
Sea $f\colon A \pto B$ una función parcial. La \emph{gráfica} de~$f$
es la relación $R_f \subseteq A \times B$ definida por
\[
R_f = \Setabs{\tuple{x,y}}{f(x) = y}.
\]
\end{defn}

\begin{prop}
Supongamos que $R \subseteq A \times B$ tiene la propiedad de que siempre que $Rxy$ y $Rxy'$, entonces $y = y'$. Entonces $R$ es la gráfica de la función parcial $f\colon X \pto Y$ definida por: si existe un $y$ tal que $Rxy$, entonces $f(x) = y$, de lo contrario $f(x) \fundefined$. Si $R$ es también \emph{serial}, es decir, para cada $x \in X$ existe un $y \in Y$ tal que $Rxy$, entonces $f$ es total.
\end{prop}

\begin{proof}
Supongamos que existe un $y$ tal que $Rxy$. Si existiera otro $y' \neq y$ tal que $Rxy'$, la condición sobre $R$ se violaría. Por lo tanto, si existe un $y$ tal que $Rxy$, ese $y$ es único, y por lo tanto $f$ está bien definida. Obviamente, $R_f = R$ y $f$ es total si~$R$ es serial.
\end{proof}

\end{document}
