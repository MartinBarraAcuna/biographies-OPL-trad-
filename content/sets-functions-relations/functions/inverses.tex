% Part: sets-functions-relations
% Chapter: functions
% Section: inverses

\documentclass[../../../include/open-logic-section]{subfiles}

\begin{document}

\olfileid{sfr}{fun}{inv}
\olsection{Inversas de Funciones}

\begin{explain}
Pensamos en las funciones como mapas. Una pregunta obvia que hacer sobre
las funciones, entonces, es si el mapeo puede ser "invertido". Por
ejemplo, la función sucesor $f(x) = x + 1$ puede ser invertida, en
el sentido de que la función $g(y) = y - 1$ "deshace" lo que $f$ hace.

Pero debemos ser cuidadosos. Aunque la definición de~$g$ define una
función $\Int \to \Int$, no define una \emph{función} $\Nat \to \Nat$, ya que $g(0) \notin \Nat$. Así que incluso en casos simples, no es del todo obvio si una función puede ser invertida; puede depender del dominio y codominio.

Esto se hace más preciso mediante la noción de una inversa de una función.
\end{explain}

\begin{defn}
Una función $g \colon B \to A$ es una \emph{inversa} de una función $f
\colon A \to B$ si $f(g(y)) = y$ y $g(f(x)) = x$ para todo $x \in A$
e $y \in B$.
\end{defn}

Si $f$ tiene una inversa~$g$, a menudo escribimos $f^{-1}$ en lugar de~$g$.

\begin{explain}
Ahora determinaremos cuándo las funciones tienen inversas. Un buen candidato
para una inversa de $f\colon A \to B$ es $g\colon B \to A$ "definida
por"
\[
g(y) = \text{"el" $x$ tal que $f(x) = y$.}
\]
Pero las comillas de miedo alrededor de "definida por" (y "el") sugieren que
esto no es una definición. Al menos, no siempre funcionará, con
completa generalidad. Porque, para que esta definición especifique una
función, tiene que haber uno y solo un~$x$ tal que $f(x) =
y$---la salida de~$g$ tiene que estar especificada de manera única. Además, tiene que estar especificada para cada $y \in B$. Si hay $x_1$ y $x_2 \in
A$ con $x_1 \neq x_2$ pero $f(x_1) = f(x_2)$, entonces $g(y)$ no estaría
especificada de manera única para $y = f(x_1) = f(x_2)$. Y si no hay ningún~$x$
en absoluto tal que $f(x) = y$, entonces $g(y)$ no está especificada en absoluto. En otras palabras, para que $g$ esté definida, $f$ debe ser tanto !!{inyectiva} como !!{sobreyectiva}.

Vamos despacio. Dividiremos la pregunta en dos: Dada una
función~$f\colon A \to B$, ¿cuándo existe una función $g\colon B \to A$
tal que $g(f(x)) = x$? Tal $g$ "deshace" lo que $f$ hace, y se
llama una \emph{inversa izquierda} de~$f$. En segundo lugar, ¿cuándo existe una
función $h\colon B \to A$ tal que $f(h(y)) = y$? Tal $h$ se
llama una \emph{inversa derecha} de~$f$---$f$ "deshace" lo que $h$ hace.
\end{explain}

\begin{prop}
Si $f\colon A \to B$ es !!{inyectiva}, entonces existe una \emph{inversa izquierda}~$g\colon B \to A$ de~$f$ tal que $g(f(x)) = x$ para todo $x \in A$.
\end{prop}

\begin{proof}
Supongamos que $f\colon A \to B$ es !!{inyectiva}. Consideremos un $y \in B$.
Si $y \in \ran{f}$, existe un $x \in A$ tal que $f(x) = y$. Debido a que $f$ es !!{inyectiva}, solo hay un tal~$x \in A$. Entonces podemos definir: $g(y) = x$, es decir, $g(y)$ es "el" $x \in A$ tal que $f(x) = y$. Si $y \notin \ran{f}$, podemos mapearlo a cualquier~$a \in A$. Así que, podemos elegir un $a \in A$ y definir $g\colon B \to A$ por:
\[
g(y) = \begin{cases}
  x & \text{si $f(x) = y$}\\
  a & \text{si $y \notin \ran{f}$.}
\end{cases}
\]
Está definida para todo $y \in B$, ya que para cada tal $y \in \ran{f}$
existe exactamente un $x \in A$ tal que $f(x) = y$. Por definición, si
$y = f(x)$, entonces $g(y) = x$, es decir, $g(f(x)) = x$.
\end{proof}

\begin{prob}
Demuestra que si $f\colon A \to B$ tiene una inversa izquierda~$g$, entonces $f$ es !!{inyectiva}.
\end{prob}

\begin{prop}
Si $f\colon A \to B$ es !!{sobreyectiva}, entonces existe una \emph{inversa derecha}~$h\colon B \to A$ de~$f$ tal que $f(h(y)) = y$ para todo~$y \in B$.
\end{prop}

\begin{proof}
Supongamos que $f\colon A \to B$ es !!{sobreyectiva}. Consideremos un $y \in B$. Dado que $f$ es !!{sobreyectiva}, existe un $x_y \in A$ con $f(x_y) = y$. Entonces podemos definir: $h(y) = x_y$, es decir, para cada $y \in B$ elegimos algún $x \in A$ tal que $f(x) = y$; dado que $f$ es !!{sobreyectiva}, siempre hay al menos uno para elegir.\footnote{Dado que $f$ es !!{sobreyectiva}, para cada~$y \in B$ el conjunto $\Setabs{x}{f(x) = y}$ no está vacío. Nuestra definición de~$h$ requiere que elijamos un solo $x$ de cada uno de estos conjuntos. Que esto sea siempre posible no es realmente obvio---la posibilidad de hacer estas elecciones simplemente se asume como un axioma. En otras palabras, esta proposición asume el llamado Axioma de Elección, un asunto que \oliflabeldef{sth:choice::chap}{revisitaremos en \olref[sth][choice][]{chap}}{pasaremos por alto}. Sin embargo, en muchos casos específicos, por ejemplo, cuando $A = \Nat$ o es finito, o cuando $f$ es !!{biyectiva}, el Axioma de Elección no es necesario. (En el caso particular cuando $f$ es !!{biyectiva}, para cada $y \in B$ el conjunto $\Setabs{x}{f(x) = y}$ tiene exactamente un !!{elemento}, por lo que no hay elección que hacer.)} Por definición, si $x = h(y)$, entonces $f(x) = y$, es decir, para cualquier $y \in B$, $f(h(y)) = y$.
\end{proof}

\begin{prob}
Demuestra que si $f\colon A \to B$ tiene una inversa derecha~$h$, entonces $f$ es !!{sobreyectiva}.
\end{prob}

\begin{explain}
Combinando las ideas de la prueba anterior, ahora obtenemos que cada !!{biyección} tiene una inversa, es decir, existe una única función que es tanto una inversa izquierda como derecha de~$f$.
\end{explain}

\begin{prop}\ollabel{prop:bijection-inverse}
Si $f\colon A \to B$ es !!{biyectiva}, existe una función~$f^{-1}\colon B \to A$ tal que para todo $x \in A$, $f^{-1}(f(x)) = x$ y para todo $y \in B$, $f(f^{-1}(y)) = y$.
\end{prop}

\begin{proof}
Ejercicio.
\end{proof}

\begin{prob}
Prueba \olref[sfr][fun][inv]{prop:bijection-inverse}. Tienes que definir~$f^{-1}$, mostrar que es una función, y mostrar que es una inversa de~$f$, es decir, $f^{-1}(f(x)) = x$ y $f(f^{-1}(y)) = y$ para todo $x \in A$ e $y \in B$.
\end{prob}

\begin{explain}
Hay una forma ligeramente más general de extraer inversas. Vimos en \olref[kin]{sec} que cada función $f$ induce !!una{sobreyección} $f' \colon A \to \ran{f}$ al dejar $f'(x) = f(x)$ para todo $x \in A$. Claramente, si $f$ es !!{inyectiva}, entonces $f'$ es !!{biyectiva}, por lo que tiene una inversa única por \olref{prop:bijection-inverse}. Por un abuso de notación muy menor, a veces llamamos a la inversa de $f'$ simplemente "la inversa de~$f$".
\end{explain}

\begin{prop}\ollabel{prop:left-right}
Demuestra que si $f\colon A \to B$ tiene una inversa izquierda~$g$ y una inversa derecha~$h$, entonces $h = g$.
\end{prop}

\begin{proof}
Ejercicio.
\end{proof}

\begin{prob}
Prueba \olref[sfr][fun][inv]{prop:left-right}.
\end{prob}

\begin{prop}\ollabel{prop:inverse-unique}
Cada función~$f$ tiene como máximo una inversa.
\end{prop}

\begin{proof}
Supongamos que $g$ y $h$ son ambas inversas de~$f$. Entonces, en particular, $g$ es una inversa izquierda de~$f$ y $h$ es una inversa derecha. Por \olref{prop:left-right}, $g = h$.
\end{proof}

\end{document}
