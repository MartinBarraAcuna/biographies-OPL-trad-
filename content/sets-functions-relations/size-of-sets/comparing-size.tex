% Part:sets-functions-relations
% Chapter: size-of-sets
% Section: comparing-sizes

\documentclass[../../../include/open-logic-section]{subfiles}

\begin{document}

\olfileid{sfr}{siz}{car}

\olsection{Sets of Different Sizes, and Cantor's Theorem}

\begin{explain}
We have offered a precise statement of the idea that two sets have the
same size. We can also offer a precise statement of the idea that one
set is smaller than another. Our definition of ``is smaller than (or
equinumerous)'' will require, instead of !!a{bijection} between the
sets, !!a{injection} from the first set to the second. If such a
function exists, the size of the first set is less than or equal to
the size of the second. Intuitively, !!a{injection} from one set to
another guarantees that the range of the function has at least as many
!!{element}s as the domain, since no two !!{element}s of the domain
map to the same !!{element} of the range.
\end{explain}

\begin{defn}
$A$ is \emph{no larger than}~$B$, written $\cardle{A}{B}$, iff there
is !!a{injection} $f \colon A \to B$.
\end{defn}

It is clear that this is a reflexive and transitive relation, but that
it is not symmetric (this is left as an exercise). We can also
introduce a notion, which states that one set is (strictly) smaller
than another. 

\begin{defn}
$A$ is \emph{smaller than}~$B$, written $\cardless{A}{B}$, iff there
is !!a{injection}~$f\colon A \to B$ but no !!{bijection}~$g\colon A
\to B$, i.e., $\cardle{A}{B}$ and $\cardneq{A}{B}$.
\end{defn}

It is clear that this relation is irreflexive
and transitive. (This is left as an exercise.) Using this notation, we
can say that a set $A$ is !!{enumerable} iff $\cardle{A}{\Nat}$, and
that $A$ is !!{nonenumerable} iff $\cardless{\Nat}{A}$. This allows us
to restate
\oliflabeldef{sfr:siz:nen-alt:thm:nonenum-pownat}{%
\olref[sfr][siz][nen-alt]{thm:nonenum-pownat}
as the observation that
$\cardless{\Nat}{\Pow{\Nat}}$}{%
\olref[sfr][siz][nen]{thm:nonenum-pownat}
as the observation that $\cardless{\PosInt}{\Pow{\PosInt}}$}. In fact,
\citet{Cantor1892} proved that this last point is \emph{perfectly
general}:

\begin{thm}[Cantor]\ollabel{thm:cantor}
$\cardless{A}{\Pow{A}}$, for any set $A$.
\end{thm}

\begin{proof}
The map $f(x) = \{x\}$ is !!a{injection} $f \colon A \to \Pow{A}$,
since if $x \neq y$, then also $\{x\} \neq \{y\}$ by extensionality,
and so $f(x) \neq f(y)$. So we have that $\cardle{A}{\Pow{A}}$. 

\begin{editorial}
We present the slow proof if \olref[nen]{sec} is
present, otherwise a faster proof matching \olref[nen-alt]{sec}.
\end{editorial}

\oliflabeldef{sfr:siz:nen:sec}{% 
  We will now show that there cannot be !!a{surjective} function~$g\colon A \to
  \Pow{A}$, let alone !!a{bijective} one, and hence that
  $\cardneq{A}{\Pow{A}}$. For suppose that $g\colon A \to \Pow{A}$.
  Since $g$ is total, every $x \in A$ is mapped to a subset $g(x)
  \subseteq A$. We can show that $g$ cannot be surjective. To do this, we
  define a subset~$\overline{A} \subseteq A$ which by definition cannot be in the
  range of~$g$. Let
  \[
  \overline{A} = \Setabs{x \in A}{x \notin g(x)}.
  \]
  Since $g(x)$ is defined for all $x \in A$, $\overline{A}$ is clearly
  a well-defined subset of~$A$.  But, it cannot be in the range
  of~$g$. Let $x \in A$ be arbitrary, we will show that $\overline{A} \neq
  g(x)$.  If $x \in g(x)$, then it does not satisfy $x \notin g(x)$,
  and so by the definition of~$\overline{A}$, we have $x \notin
  \overline{A}$.  If $x \in \overline{A}$, it must satisfy the
  defining property of~$\overline{A}$, i.e., $x \in A$ and $x \notin
  g(x)$. Since $x$ was arbitrary, this shows that for each $x \in
  \overline{A}$, $x \in g(x)$ iff $x \notin \overline{A}$, and so
  $g(x) \neq \overline{A}$.  In other words, $\overline{A}$ cannot be
  in the range of~$g$, contradicting the assumption that~$g$ is
  surjective.}{It remains to show that $\cardneq{A}{\Pow{A}}$. For
  reductio, suppose $\cardeq{A}{\Pow{A}}$, i.e., there is some
  !!{bijection} $g \colon A \to \Pow{A}$. Now consider:
  \[
    D = \Setabs{x \in A}{x \notin g(x)}
  \]
  Note that $D \subseteq A$, so that $D \in \Pow{A}$. Since $g$ is
  !!a{bijection}, there is some $y \in A$ such that $g(y) = D$. But
  now we have:
  \[
    y \in g(y) \text{ iff } y \in D \text{ iff } y \notin g(y).
  \]
  This is a contradiction; so $\cardneq{A}{\Pow{A}}$.}{}
\end{proof}

\begin{explain}
\oliflabeldef{sfr:siz:nen:thm:nonenum-pownat}{It's instructive to
  compare the proof of \olref{thm:cantor} to that of
  \olref[nen]{thm:nonenum-pownat}. There we showed that for any list
  $Z_1$, $Z_2$, \dots, of subsets of~$\PosInt$ one can construct a
  set~$\overline{Z}$ of numbers guaranteed not to be on the list. It
  was guaranteed not to be on the list because, for every $n \in
  \PosInt$, $n \in Z_n$ iff $n \notin \overline{Z}$. This way, there
  is always some number that is !!a{element} of one of $Z_n$ or
  $\overline{Z}$ but not the other. We follow the same idea here,
  except the indices~$n$ are now !!{element}s of~$A$ instead
  of~$\PosInt$. The set $\overline{B}$ is defined so that it is
  different from~$g(x)$ for each $x \in A$, because $x \in g(x)$ iff
  $x \notin \overline{B}$. Again, there is always !!a{element} of~$A$
  which is !!a{element} of one of $g(x)$ and $\overline{B}$ but not
  the other. And just as $\overline{Z}$ therefore cannot be on the
  list $Z_1$, $Z_2$, \dots, $\overline{B}$ cannot be in the range
  of~$g$.}{}

\oliflabeldef{sfr:siz:nen-alt:thm:nonenum-pownat}{It's instructive to
  compare the proof of \olref{thm:cantor} to that of
  \olref[nen-alt]{thm:nonenum-pownat}. There we showed that for any
  list $N_0$, $N_1$, $N_2$, \dots, of subsets of~$\Nat$ we can construct a
  set~$D$ of numbers guaranteed not to be on the list. It was
  guaranteed not to be on the list because $n \in N_n$ iff $n \notin
  D$, for every $n \in \Nat$. We follow the same idea here, except the
  indices~$n$ are now !!{element}s of~$A$ rather than of~$\Nat$. The
  set $D$ is defined so that it is different from~$g(x)$ for each $x
  \in A$, because $x \in g(x)$ iff $x \notin D$.}{}

The proof is also worth comparing with the proof of Russell's Paradox,
\olref[sfr][set][rus]{thm:russells-paradox}. Indeed, Cantor's Theorem was
the inspiration for Russell's own paradox.
\end{explain}

\begin{prob}
  Show that there cannot be !!a{injection} $g\colon \Pow{A} \to
  A$, for any set~$A$. Hint: Suppose $g\colon \Pow{A} \to A$ is
  !!{injective}. Consider $D = \Setabs{g(B)}{B \subseteq A \text{ and
  } g(B) \notin B}$. Let $x = g(D)$. Use the fact that $g$ is
  !!{injective} to derive a contradiction. 
\end{prob}

\end{document}
