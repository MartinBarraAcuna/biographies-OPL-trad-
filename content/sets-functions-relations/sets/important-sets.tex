% Part: sets-functions-relations
% Chapter: sets
% Section: important-sets

\documentclass[../../../include/open-logic-section]{subfiles}

\begin{document}

\olfileid{sfr}{set}{imp}
\olsection{Algunos conjuntos importantes}

\begin{ex}
Principalmente trataremos con conjuntos cuyos !!{elementos} son
objetos matemáticos. Cuatro de estos conjuntos son lo suficientemente
importantes como para tener nombres específicos:
\begin{multline*}
  \Nat = \{0, 1, 2, 3, \ldots\} \\
  \shoveright{\text{el conjunto de los números naturales}}\\
  \shoveleft{\Int = \{\ldots, -2, -1, 0, 1, 2, \ldots\}} \\
  \shoveright{\text{el conjunto de los números enteros}}\\
  \shoveleft{\Rat = \Setabs{\nicefrac{m}{n}}{m, n \in \Int\text{ y }n \neq 0}}\\
  \shoveright{\text{el conjunto de los números racionales}}\\
  \shoveleft{\Real = (-\infty, \infty)}\\
  \text{el conjunto de los números reales (el continuo)}
\end{multline*}
Todos estos son conjuntos \emph{infinitos}, es decir, cada uno tiene
infinitamente muchos !!{elementos}.

A medida que avanzamos a través de estos conjuntos, estamos añadiendo
\emph{más} números a nuestro repertorio. De hecho, debería ser claro que $\Nat \subseteq \Int
\subseteq \Rat \subseteq \Real$: después de todo, cada número natural es un
entero; cada entero es un racional; y cada racional es un real. Igualmente, debería ser claro que $\Nat \subsetneq \Int \subsetneq
\Rat$, ya que $-1$ es un entero pero no un número natural, y
$\nicefrac{1}{2}$ es racional pero no entero. Es menos obvio
que $\Rat \subsetneq \Real$, es decir, que hay algunos números reales que no son racionales\oliflabeldef{sfr:arith:real:realline}{, pero volveremos a esto en \olref[arith][real]{realline}}{}.

A veces también usaremos el conjunto de los enteros positivos $\PosInt = \{1,
2, 3, \dots\}$ y el conjunto que contiene solo los dos primeros números naturales $\Bin = \{0, 1\}$.
\end{ex}

\begin{tagblock}{compsci}
\begin{ex}[Cadenas]
Otro ejemplo interesante es el conjunto $A^{*}$ de \emph{cadenas finitas} sobre un alfabeto $A$: cualquier secuencia finita de elementos de~$A$ es una cadena sobre $A$. Incluimos la \emph{cadena vacía $\Lambda$} entre las cadenas sobre~$A$, para cada alfabeto~$A$. Por ejemplo,
\begin{multline*}
\Bin^*
=\{\Lambda,0,1,00,01,10,11,\\
000,001,010,011,100,101,110,111,0000,\ldots\}.
\end{multline*}
Si $x=x_{1}\ldots x_{n}\in A^{*}$ es una cadena que consta de $n$
"letras" de $A$, entonces decimos que la \emph{longitud} de la cadena es~$n$ y escribimos $\len{x}=n$.
\end{ex}
\end{tagblock}

\begin{ex}[Secuencias infinitas]
Para cualquier conjunto $A$ también podemos considerar el conjunto~$A^\omega$ de secuencias infinitas de !!{elementos} de~$A$. Una secuencia infinita $a_1a_2a_3a_4\dots$ consiste en una lista infinita unidireccional de objetos, cada uno de los cuales es !!un{elemento} de~$A$.
\end{ex}

\end{document}
