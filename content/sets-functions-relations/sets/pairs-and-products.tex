% Part: sets-functions-relations
% Chapter: sets
% Section: pairs-and-products

\documentclass[../../../include/open-logic-section]{subfiles}

\begin{document}

\olfileid{sfr}{set}{pai}
\olsection{Pares, Tuplas y Productos Cartesianos}

\begin{explain}
De la extensionalidad se deduce que los conjuntos no tienen un orden en sus elementos. Por lo tanto, si queremos representar el orden, utilizamos \emph{pares ordenados} $\tuple{x, y}$. En un par no ordenado $\{x, y\}$, el orden no importa: $\{x, y\} = \{y, x\}$. En un par ordenado, sí importa: si $x \neq y$, entonces $\tuple{x, y} \neq \tuple{y, x}$.

¿Cómo debemos pensar en los pares ordenados en la teoría de conjuntos? Fundamentalmente, queremos preservar la idea de que los pares ordenados son idénticos si y solo si comparten el mismo primer elemento y comparten el mismo segundo elemento, es decir:
\[
  \tuple{a, b}= \tuple{c, d}\text{ si y solo si tanto }a = c \text{ como }b=d.
\]
Podemos definir pares ordenados en la teoría de conjuntos utilizando la definición de Wiener--Kuratowski.
\end{explain}

\begin{defn}[Par ordenado]\ollabel{wienerkuratowski}
	$\tuple{a, b} = \{\{a\}, \{a, b\}\}$.
\end{defn}

\begin{prob}
	Usando \olref[sfr][set][pai]{wienerkuratowski}, prueba que $\tuple{a, b}= \tuple{c, d}$ si y solo si tanto $a = c$ como $b=d$.
\end{prob}

\begin{explain}
Habiendo fijado una definición de par ordenado, podemos usarla para definir conjuntos adicionales. Por ejemplo, a veces también queremos secuencias ordenadas de más de dos objetos, por ejemplo, \emph{triplas} $\tuple{x, y, z}$, \emph{cuádruplas} $\tuple{x, y, z, u}$, y así sucesivamente.  Podemos pensar en las triplas como pares ordenados especiales, donde el primer elemento es en sí mismo un par ordenado: $\tuple{x, y, z}$ es $\tuple{\tuple{x, y},z}$. Lo mismo ocurre con las cuádruplas: $\tuple{x,y,z,u}$ es $\tuple{\tuple{\tuple{x,y},z},u}$, y así sucesivamente. En general, hablamos de \emph{$n$-tuplas ordenadas} $\tuple{x_1, \dots, x_n}$.

Ciertos conjuntos de pares ordenados, u otras $n$-tuplas ordenadas, serán útiles.
\end{explain}

\begin{defn}[Producto cartesiano]
Dados los conjuntos $A$ y $B$, su \emph{producto cartesiano} $A \times B$ se define por
\[
  A \times B = \Setabs{\tuple{x, y}}{x \in A \text{ y } y \in B}.
\]
\end{defn}

\begin{ex}
Si $A = \{0, 1\}$, y $B = \{1, a, b\}$, entonces su producto es
\[
A \times B = \{ \tuple{0, 1}, \tuple{0, a}, \tuple{0, b},
    \tuple{1, 1}, \tuple{1, a}, \tuple{1, b} \}.
\]
\end{ex}

\begin{ex}
Si $A$ es un conjunto, el producto de $A$ consigo mismo, $A \times A$, también se escribe~$A^2$. Es el conjunto de \emph{todos} los pares $\tuple{x, y}$ con $x, y \in A$. El conjunto de todas las triplas $\tuple{x, y, z}$ es $A^3$, y así sucesivamente. Podemos dar una definición recursiva:
\begin{align*}
  A^1 & = A\\
  A^{k+1} & = A^k \times A
\end{align*}
\end{ex}

\begin{prob}
Enumera todos los elementos de $\{1, 2, 3\}^3$.
\end{prob}

\begin{prop}\ollabel{cardnmprod}
Si $A$ tiene $n$ elementos y $B$ tiene $m$ elementos, entonces $A \times B$ tiene $n\cdot m$ elementos.
\end{prop}

\begin{proof}
Por cada elemento~$x$ en~$A$, hay $m$ elementos de la forma $\tuple{x, y} \in A \times B$. Sea $B_x = \Setabs{\tuple{x, y}}{y \in B}$. Dado que siempre que $x_1 \neq x_2$, $\tuple{x_1, y} \neq \tuple{x_2, y}$, $B_{x_1} \cap B_{x_2} = \emptyset$. Pero si $A = \{x_1, \dots, x_n\}$, entonces $A \times B = B_{x_1} \cup \dots \cup B_{x_n}$, y por lo tanto tiene $n\cdot m$ elementos.

Para visualizar esto, disponga los elementos de~$A \times B$ en una cuadrícula:
\[
\begin{array}{rcccc}
  B_{x_1} = & \{\tuple{x_1, y_1} & \tuple{x_1, y_2} & \dots & \tuple{x_1, y_m}\}\\
  B_{x_2} = & \{\tuple{x_2, y_1} & \tuple{x_2, y_2} & \dots & \tuple{x_2, y_m}\}\\
  \vdots & & \vdots\\
  B_{x_n} = & \{\tuple{x_n, y_1} & \tuple{x_n, y_2} & \dots & \tuple{x_n, y_m}\}
\end{array}
\]
Dado que las $x_i$ son todas diferentes, y las $y_j$ son todas diferentes, no hay dos pares en esta cuadrícula que sean iguales, y hay $n\cdot m$ de ellos.
\end{proof}

\begin{prob}
Muestra, por inducción en~$k$, que para todo $k \ge 1$, si $A$ tiene $n$ elementos, entonces $A^k$ tiene $n^k$ elementos.
\end{prob}

\begin{ex}
Si $A$ es un conjunto, una \emph{palabra} sobre~$A$ es cualquier secuencia de elementos de~$A$. Una secuencia puede considerarse como una $n$-tupla de elementos de~$A$. Por ejemplo, si $A = \{a, b, c\}$, entonces la secuencia ``$bac$'' puede considerarse como la tripla~$\tuple{b, a, c}$. Las palabras, es decir, las secuencias de símbolos, son de crucial importancia en la informática. Por convención, contamos los elementos de~$A$ como secuencias de longitud~$1$, y $\emptyset$ como la secuencia de longitud~$0$. El conjunto de \emph{todas} las palabras sobre~$A$ entonces es
\[
A^* = \{\emptyset\} \cup A \cup A^2 \cup A^3 \cup \dots
\]
\end{ex}

\end{document}
