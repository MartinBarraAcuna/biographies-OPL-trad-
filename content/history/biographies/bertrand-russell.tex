% Part: history
% Chapter: biographies
% Section: bertrand-russell

\documentclass[../../../include/open-logic-section]{subfiles}

\begin{document}

\olfileid{his}{bio}{rus} 

\olsection{Bertrand Russell}

\olphoto{russell-bertrand}{Bertrand Russell}

Bertrand Russell es aclamado como uno de los fundadores de la filosofía analítica moderna. Nacido el 18 de mayo de 1872, Russell no solo fue conocido por su trabajo en filosofía y lógica, sino que escribió muchos libros populares en diversas áreas temáticas. También fue un ardiente activista político a lo largo de su vida.

Russell nació en Trellech, Monmouthshire, Gales. Sus padres eran miembros de la nobleza británica. Eran librepensadores, e incluso se hicieron amigos de los radicales en Boston en ese momento. Desafortunadamente, los padres de Russell murieron cuando él era joven, y Russell fue enviado a vivir con sus abuelos. Allí, recibió una educación religiosa (algo que sus padres habían querido evitar a toda costa). Su abuela era muy estricta en todos los asuntos de moralidad. Durante la adolescencia, fue educado en casa principalmente por tutores privados.

La influencia de Russell en la filosofía analítica, y especialmente en la lógica, es tremenda. Estudió matemáticas y filosofía en el Trinity College, Cambridge, donde fue influenciado por el matemático y filósofo Alfred North Whitehead. En 1910, Russell y Whitehead publicaron el primer volumen de \emph{Principia Mathematica}, donde defendieron la opinión de que las matemáticas son reducibles a la lógica. Continuó publicando cientos de libros, ensayos y panfletos políticos. En 1950, ganó el Premio Nobel de Literatura.

Russell estuvo profundamente arraigado en la política y el activismo social. Durante la Primera Guerra Mundial, fue arrestado y enviado a prisión durante seis meses debido a actividades pacifistas y protestas. Mientras estaba en prisión, pudo escribir y leer, y afirma haber encontrado la experiencia "bastante agradable". Siguió siendo pacifista durante toda su vida, y fue nuevamente encarcelado por asistir a una manifestación de desarme nuclear en 1961. También sobrevivió a un accidente aéreo en 1948, donde los únicos supervivientes fueron los que estaban sentados en la sección de fumadores. Como tal, Russell afirmó que debía su vida al tabaco. Russell se casó cuatro veces, pero tenía reputación de tener aventuras extramatrimoniales. Murió el 2 de febrero de 1970 a la edad de 97 años en Penrhyndeudraeth, Gales.

\begin{reading}
Russell escribió una autobiografía en tres partes, que abarca su vida desde 1872 hasta 1967 \citep{Russell1967,Russell1968,Russell1969}. El Centro de Investigación Bertrand Russell en la Universidad McMaster alberga los archivos de Bertrand Russell. Consulte su sitio web en \citet{Duncan2015}, para obtener información sobre los volúmenes de sus obras completas (incluidos índices de búsqueda) y proyectos de archivo. El artículo de Russell \emph{Sobre la denotación} \citep{Russell1905} es un clásico de la filosofía analítica del siglo XX.

La entrada de la Enciclopedia de Filosofía de Stanford sobre Russell \citep{Irvine2015} tiene clips de sonido de Russell hablando sobre el Deseo y la teoría política. Muchos videos de entrevistas con Russell están disponibles en línea. Para verlo hablar sobre fumar y estar involucrado en un accidente aéreo, por ejemplo, véase \citet{RussellND}. Algunas de las obras de Russell, incluida su \emph{Introducción a la filosofía matemática}, están disponibles como audiolibros gratuitos en \citet{LibriVoxND}.
\end{reading}

\end{document}
