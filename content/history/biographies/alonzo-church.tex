% Part: history
% Chapter: biographies 
% Section: alonzo-church

\documentclass[../../../include/open-logic-section]{subfiles}

\begin{document}

\olfileid{his}{bio}{chu}

\olsection{Alonzo Church}

\olphoto{church-alonzo}{Alonzo Church}

Alonzo Church nació en Washington, DC el 14 de junio de 1903. En su primera infancia, un incidente con una pistola de aire comprimido dejó a Church ciego de un ojo. Terminó la escuela preparatoria en Connecticut en 1920 y comenzó su educación universitaria en Princeton ese mismo año. Completó sus estudios de doctorado en 1927. Después de un par de años en el extranjero, Church regresó a Princeton. Church era conocido por ser extremadamente educado y cuidadoso. Su escritura en la pizarra era inmaculada, y conservaba documentos importantes cubriéndolos cuidadosamente con cemento Duco (un pegamento transparente). Fuera de sus actividades académicas, disfrutaba leyendo revistas de ciencia ficción y no dudaba en escribir a los editores si detectaba alguna imprecisión en la escritura.

Los logros académicos de Church fueron grandes. Junto con sus estudiantes Stephen Kleene y Barkley Rosser, desarrolló una teoría de calculabilidad efectiva, el cálculo lambda, independientemente del desarrollo de la máquina de Turing por Alan Turing. Las dos definiciones de computabilidad son equivalentes, y dan lugar a lo que ahora se conoce como la \emph{Tesis de Church-Turing}, que establece que una función de los números naturales es efectivamente computable si y solo si es computable mediante una máquina de Turing (o cálculo lambda). También demostró lo que ahora se conoce como el \emph{Teorema de Church}: El problema de decisión para la validez de las fórmulas de primer orden es irresoluble.

Church continuó su trabajo hasta una edad avanzada. En 1967 dejó Princeton para ir a UCLA, donde fue profesor hasta su jubilación en 1990. Church falleció el 1 de agosto de 1995 a la edad de 92 años.

\begin{reading}
Para una breve biografía de Church, véase \citet{EndertonND}. Los escritos originales de Church sobre el cálculo lambda y el Entscheidungsproblem (Tesis de Church) son \citet{Church1936,Church1936a}. \citet{Aspray1984} registra una entrevista con Church sobre la comunidad matemática de Princeton en la década de 1930. Church escribió una serie de reseñas de libros del \emph{Journal of Symbolic Logic} desde 1936 hasta 1979. Todos están archivados en el sitio web de John MacFarlane \citep{MacFarlane2015}.
\end{reading}
\end{document}
