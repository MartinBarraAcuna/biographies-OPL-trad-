% Part: history
% Chapter: biographies 
% Section: alfred-tarski

\documentclass[../../../include/open-logic-section]{subfiles}

\begin{document}

\olfileid{his}{bio}{tar}

\olsection{Alfred Tarski}

\olphoto{tarski-alfred}{Alfred Tarski}

Alfred Tarski nació el 14 de enero de 1901 en Varsovia, Polonia (entonces parte del Imperio Ruso). Descrito como "napoleónico", Tarski era bullicioso, hablador e intenso. Su energía se reflejaba a menudo en sus conferencias: una vez prendió fuego a una papelera mientras se deshacía de un cigarrillo durante una conferencia, y se le prohibió volver a dar conferencias en ese edificio.

Tarski tenía sed de conocimiento desde muy joven. Aunque más tarde en su vida les diría a sus estudiantes que estudió lógica porque era la única clase en la que obtuvo una B, sus registros de la escuela secundaria muestran que obtuvo A en todas las materias, incluso en lógica. Estudió en la Universidad de Varsovia de 1918 a 1924. Tarski inicialmente tenía la intención de estudiar biología, pero se interesó en matemáticas, filosofía y lógica, ya que la universidad era el centro de la Escuela de Lógica y Filosofía de Varsovia. Tarski obtuvo su doctorado en 1924 bajo la supervisión de Stanis\l{}aw Le\'{s}niewski.

Antes de emigrar a los Estados Unidos en 1939, Tarski completó algunos de sus trabajos más importantes mientras trabajaba como profesor de escuela secundaria en Varsovia. Su trabajo sobre consecuencia lógica y verdad lógica fue escrito durante este tiempo. En 1939, Tarski estaba visitando los Estados Unidos para una gira de conferencias. Durante su visita, Alemania invadió Polonia, y debido a su herencia judía, Tarski no pudo regresar. Su esposa e hijos permanecieron en Polonia hasta el final de la guerra, pero luego pudieron emigrar a los Estados Unidos también. Tarski enseñó en Harvard, el College of the City of New York y el Instituto de Estudios Avanzados en Princeton, y finalmente en la Universidad de California, Berkeley. Allí fundó el programa multidisciplinario en Lógica y Metodología de la Ciencia. Tarski murió el 26 de octubre de 1983 a la edad de 82 años.

\begin{reading}
Para obtener más información sobre la vida de Tarski, véase la biografía \emph{Alfred Tarski: Vida y Lógica} \citep{Feferman2004}. Los trabajos seminales de Tarski sobre consecuencia lógica y verdad están disponibles en inglés en \citep{Tarski1983}. Todos los trabajos originales de Tarski han sido recopilados en una serie de cuatro volúmenes, \citep{Tarski1981}.
\end{reading}

\end{document}
