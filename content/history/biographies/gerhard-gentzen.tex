% Part: history
% Chapter: biographies
% Section: gerhard-gentzen

\documentclass[../../../include/open-logic-section]{subfiles}

\begin{document}

\olfileid{his}{bio}{gen}

    \olsection{Gerhard Gentzen}

\olphoto{gentzen-gerhard}{Gerhard Gentzen}

Gerhard Gentzen es conocido principalmente como el creador de la teoría estructural de la prueba, y específicamente la creación de los sistemas de !!{derivación} de deducción natural y cálculo de secuentes. Nació el 24 de noviembre de 1909 en Greifswald, Alemania. Gerhard fue educado en casa durante tres años antes de asistir a la escuela preparatoria, donde estaba por detrás de la mayoría de sus compañeros en términos de educación. A pesar de esto, fue un estudiante brillante y mostró una gran aptitud para las matemáticas. Sus intereses eran variados, y, por ejemplo, también escribió poemas para su madre y obras de teatro para el teatro escolar.

Gentzen comenzó sus estudios universitarios en la Universidad de Greifswald, pero se trasladó a G\"{o}ttingen, Múnich y Berlín. Recibió su doctorado en 1933 de la Universidad de G\"{o}ttingen bajo la dirección de Hermann Weyl. (Paul Bernays supervisó la mayor parte de su trabajo, pero fue despedido de la universidad por los nazis.) En 1934, Gentzen comenzó a trabajar como asistente de David Hilbert. Ese mismo año desarrolló los sistemas de cálculo de secuentes y !!{derivación} de deducción natural, en sus artículos \emph{Untersuchungen \"{u}ber das logische Schlie\ss en I--II [Investigaciones sobre la deducción lógica I--II]}. Demostró la consistencia de los axiomas de Peano en 1936.

La relación de Gentzen con los nazis es complicada. Al mismo tiempo que su mentor Bernays fue obligado a abandonar Alemania, Gentzen se unió a la rama universitaria de la SA, la organización paramilitar nazi. Como muchos alemanes, fue miembro del partido nazi. Durante la guerra, sirvió como oficial de telecomunicaciones para la unidad de inteligencia aérea. Sin embargo, en 1942 fue relevado de sus funciones debido a una crisis nerviosa. No está claro si la lealtad de Gentzen estaba con el partido nazi, o si se unió al partido para asegurar el éxito académico.

En 1943, a Gentzen se le ofreció un puesto académico en el Instituto Matemático de la Universidad Alemana de Praga, que aceptó. Sin embargo, en 1945 los ciudadanos de Praga se rebelaron contra la ocupación alemana. Las fuerzas soviéticas llegaron a la ciudad y arrestaron a todos los profesores de la universidad. Debido a su membresía en organizaciones nazis, Gentzen fue llevado a un campo de trabajos forzados. Murió de desnutrición en su celda el 4 de agosto de 1945 a la edad de 35 años.

\begin{reading}
Para una biografía completa de Gentzen, véase \citet{Menzler-Trott2007}. Una lectura interesante sobre los matemáticos bajo el régimen nazi, que ofrece una breve nota sobre la vida de Gentzen, es proporcionada por \citet{Segal2014}. Los artículos de Gentzen sobre la deducción lógica están disponibles en el alemán original \citep{Gentzen1935a,Gentzen1935b}. Las traducciones al inglés de los artículos de Gentzen han sido recopiladas en un solo volumen por \citet{Gentzen1969}, que también incluye un esbozo biográfico.
\end{reading}

\end{document}
