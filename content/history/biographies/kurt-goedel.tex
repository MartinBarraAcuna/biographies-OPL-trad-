% Part: history
% Chapter: biographies 
% Section: kurt-goedel

\documentclass[../../../include/open-logic-section]{subfiles}

\begin{document}

\olfileid{his}{bio}{god}

\olsection{Kurt G\"odel}

\olphoto{goedel-kurt}{Kurt G{\"o}del}

Kurt G{\"o}del (\textsc{ger}-dle) nació el 28 de abril de 1906 en Br{\"u}nn en el imperio austrohúngaro (ahora Brno en la República Checa). Debido a su naturaleza inquisitiva y brillante, el joven Kurtele a menudo era llamado "Der kleine Herr Warum" (El pequeño señor Por qué) por su familia. Sobresalió en los estudios académicos desde la escuela primaria en adelante, donde obtuvo menos de la calificación más alta solo en matemáticas. G{\"o}del a menudo estaba ausente de la escuela debido a su mala salud y estaba exento de educación física. Fue diagnosticado con fiebre reumática durante su infancia. A lo largo de su vida, creyó que esto afectó permanentemente su corazón a pesar de que la evaluación médica decía lo contrario.

G{\"o}del comenzó a estudiar en la Universidad de Viena en 1924 y completó sus estudios de doctorado en 1929. Inicialmente tenía la intención de estudiar física, pero sus intereses pronto se trasladaron a las matemáticas y especialmente a la lógica, en parte debido a la influencia del filósofo Rudolf Carnap. Su disertación, escrita bajo la supervisión de Hans Hahn, demostró el teorema de completitud de la lógica de predicados de primer orden con identidad \citep{Godel1929}. Solo un año después, obtuvo sus resultados más famosos: el primer y segundo teorema de incompletitud (publicados en \citealt{Godel1931}). Durante su tiempo en Viena, G{\"o}del estuvo muy involucrado con el Círculo de Viena, un grupo de filósofos de mentalidad científica que incluía a Carnap, cuyo trabajo fue especialmente influenciado por los resultados de G{\"o}del.

En 1938, G\"odel se casó con Adele Nimbursky. Sus padres no estaban contentos: no solo era seis años mayor que él y ya divorciada, sino que trabajaba como bailarina en un club nocturno. Sin embargo, las presiones sociales no afectaron a G{\"o}del, y permanecieron felizmente casados hasta su muerte.

Después de que la Alemania nazi se anexionó Austria en 1938, G{\"o}del y Adele emigraron a los Estados Unidos, donde asumió un puesto en el Instituto de Estudios Avanzados en Princeton, Nueva Jersey. A pesar de su introversión y naturaleza excéntrica, el tiempo de G{\"o}del en Princeton fue colaborativo y fructífero. Publicó ensayos sobre teoría de conjuntos, filosofía y física. En particular, entabló una fuerte amistad con su colega en el IAS, Albert Einstein.

En sus últimos años, la salud mental de G{\"o}del se deterioró. La hospitalización de su esposa en 1977 significó que ya no podía cocinar sus comidas para él. Habiendo sufrido problemas de salud mental a lo largo de su vida, sucumbió a la paranoia. Mortalmente temeroso de ser envenenado, G{\"o}del se negó a comer. Murió de inanición el 14 de enero de 1978, en Princeton.

\begin{reading}
Para una biografía completa de la vida de G{\"o}del, véase \citet{Dawson1997}. Para otras piezas biográficas, así como ensayos sobre las contribuciones de G{\"o}del a la lógica y la filosofía, véase \citet{Wang1990}, \citet{Baaz2011}, \citet{Takeuti2003} y \citet{Sigmund2007}.

La tesis doctoral de G{\"o}del está disponible en el alemán original \citep{Godel1929}. El texto original de los teoremas de incompletitud es \citep{Godel1931}. Todos los escritos publicados e inéditos de G\"odel, así como una selección de correspondencia, están disponibles en inglés en sus \emph{Collected Papers} \citet{Godel1986,Godel1990}.

Para un tratamiento detallado de los teoremas de incompletitud de G{\"o}del, véase \citet{Smith2013}. Para una discusión informal y filosófica de los teoremas de G{\"o}del, véase el podcast de Mark Linsenmayer \citep{Linsenmayer2014}.
\end{reading}

\end{document}
