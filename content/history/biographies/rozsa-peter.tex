% Part: history
% Chapter: biographies
% Section: rozsa-peter

\documentclass[../../../include/open-logic-section]{subfiles}

\begin{document}

\olfileid{his}{bio}{pet} 

\olsection{R\'ozsa P\'eter}

\olphoto{peter-rozsa}{R\'ozsa P\'eter}

R\'ozsa P\'eter nació como R\'osza Politzer, en Budapest, Hungría, el 17 de febrero de 1905. Es mejor conocida por su trabajo en funciones recursivas, que fue esencial para la creación del campo de la teoría de la recursión.

P\'eter creció durante tiempos políticos difíciles —la Primera Guerra Mundial azotó cuando era adolescente—, pero pudo asistir a la adinerada Escuela de Niñas Maria Terezia en Budapest, de donde se graduó en 1922. Luego estudió en la Universidad P\'azm\'any P\'eter (más tarde renombrada Universidad Lor\'and E\"otv\"os) en Budapest. Comenzó a estudiar química por insistencia de su padre, pero luego cambió a matemáticas y se graduó en 1927. Aunque tenía las credenciales para enseñar matemáticas en la escuela secundaria, la situación económica en ese momento era terrible, ya que la Gran Depresión afectó la economía mundial. Durante este tiempo, P\'eter realizó trabajos ocasionales como tutora y profesora privada de matemáticas. Finalmente regresó a la universidad para realizar estudios de posgrado en matemáticas. Originalmente había planeado trabajar en teoría de números, pero después de descubrir que sus resultados ya habían sido probados, casi abandonó las matemáticas por completo. Fue alentada a trabajar en los teoremas de incompletitud de G\"odel, e inconscientemente probó varios de sus resultados de diferentes maneras. Esto restauró su confianza, y P\'eter pasó a escribir sus primeros artículos sobre teoría de la recursión, inspirada por el programa fundacional de David Hilbert. Recibió su doctorado en 1935, y en 1937 se convirtió en editora del \emph{Journal of Symbolic Logic}.

Los primeros artículos de P\'eter son ampliamente acreditados como contribuciones fundacionales al campo de la teoría de funciones recursivas. En \cite{Peter1935a}, investigó la relación entre diferentes tipos de recursión. En \cite{Peter1935b}, demostró que cierta función definida recursivamente no es recursiva primitiva. Esto simplificó un resultado anterior debido a Wilhelm Ackermann. La función simplificada de P\'eter es lo que ahora a menudo se llama la función de Ackermann —y a veces, más apropiadamente, la función de Ackermann-P\'eter. Escribió el primer libro sobre teoría de funciones recursivas \citep{Peter1951}.

A pesar de la importancia e influencia de su trabajo, P\'eter no obtuvo un puesto de enseñanza a tiempo completo hasta 1945. Durante la ocupación nazi de Hungría durante la Segunda Guerra Mundial, a P\'eter no se le permitió enseñar debido a las leyes antisemitas. En 1944, el gobierno creó un gueto judío en Budapest; el gueto fue aislado del resto de la ciudad y vigilado por guardias armados. P\'eter se vio obligada a vivir en el gueto hasta 1945, cuando fue liberado. Luego pasó a enseñar en el Colegio de Formación de Maestros de Budapest, y desde 1955 en adelante en la Universidad E\"otv\"os Lor\'and. Fue la primera matemática húngara en convertirse en Doctor Académico de Matemáticas, y la primera mujer en ser elegida miembro de la Academia de Ciencias de Hungría.

P\'{e}ter era conocida como una apasionada profesora de matemáticas, que prefería explorar la naturaleza y la belleza de los problemas matemáticos con sus estudiantes en lugar de simplemente dar conferencias. Como resultado, sus estudiantes la llamaban afectuosamente "Tía Rosa". P\'eter murió en 1977 a la edad de 71 años.

\begin{reading}
Para más lecturas biográficas, véase \citep{Oconnor2014} y \citep{Andrasfai1986}. \citet{Tamassy1994} realizó una breve entrevista con P\'eter. Para una lectura divertida sobre matemáticas, véase el libro de P\'eter \emph{Jugando con el Infinito} \citep{Peter2010}.
\end{reading}

\end{document}
