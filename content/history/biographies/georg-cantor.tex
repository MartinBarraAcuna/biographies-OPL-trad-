% Part: history
% Chapter: biographies 
% Section: georg-cantor

\documentclass[../../../include/open-logic-section]{subfiles}

\begin{document}

\olfileid{his}{bio}{can}

\olsection{Georg Cantor}

\olphoto{cantor-georg}{Georg Cantor}

Una biografía temprana de Georg Cantor (\textsc{gay}-org \textsc{kahn}-tor) afirmaba que nació y fue encontrado en un barco que navegaba hacia San Petersburgo, Rusia, y que sus padres eran desconocidos. Sin embargo, esto no es cierto; aunque nació en San Petersburgo en 1845.

Cantor recibió su doctorado en matemáticas en la Universidad de Berlín en 1867. Es conocido por su trabajo en teoría de conjuntos, y se le atribuye la fundación de la teoría de conjuntos como una disciplina de investigación distintiva. Fue el primero en demostrar que existen conjuntos infinitos de diferentes tamaños. Sus teorías, y especialmente su teoría de los infinitos, causaron mucho debate entre los matemáticos de la época, y su trabajo fue controvertido.

Las creencias religiosas de Cantor y su trabajo matemático estaban inextricablemente ligados; incluso afirmó que la teoría de los números transfinitos le había sido comunicada directamente por Dios. En su vida posterior, Cantor sufrió de enfermedad mental. A partir de 1894, y con mayor frecuencia hacia sus últimos años, Cantor fue hospitalizado. La fuerte crítica a su trabajo, incluyendo una pelea con el matemático Leopold Kronecker, llevó a la depresión y a la falta de interés en las matemáticas. Durante los episodios depresivos, Cantor se dedicaba a la filosofía y la literatura, e incluso publicó una teoría de que Francis Bacon era el autor de las obras de Shakespeare.

Cantor murió el 6 de enero de 1918, en un sanatorio en Halle.

\begin{reading}
Para biografías completas de Cantor, véase \citet{Dauben1990} y \citet{Grattan-Guinness1971}. Las opiniones radicales de Cantor también se describen en el programa de BBC Radio 4 \emph{A Brief History of Mathematics} \citep{Sautoy2014}. Si desea escuchar sobre las teorías de Cantor en forma de rap, véase \citet{Rose2012}.
\end{reading}

\end{document}
