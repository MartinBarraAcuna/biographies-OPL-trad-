% Part: history
% Chapter: biographies 
% Section: emmy-noether

\documentclass[../../../include/open-logic-section]{subfiles}

\begin{document}

\olfileid{his}{bio}{noe}

\olsection{Emmy Noether}

\olphoto{noether-emmy}{Emmy Noether}

Emmy Noether (\textsc{ner}-ter) nació en Erlangen, Alemania, el 23 de marzo de 1882, en una familia académica de clase media alta. Aclamada como la "madre del álgebra moderna", Noether realizó contribuciones innovadoras tanto a las matemáticas como a la física, a pesar de las importantes barreras a la educación de las mujeres. En Alemania en ese momento, se suponía que las jóvenes debían ser educadas en artes y no se les permitía asistir a escuelas preparatorias universitarias. Sin embargo, después de asistir como oyente a clases en las Universidades de G\"{o}ttingen y Erlangen (donde su padre era profesor de matemáticas), Noether finalmente pudo inscribirse como estudiante en Erlangen en 1904, cuando su política fue actualizada para permitir estudiantes mujeres. Recibió su doctorado en matemáticas en 1907.

A pesar de sus cualificaciones, Noether experimentó mucha resistencia durante su carrera. De 1908 a 1915, enseñó en Erlangen sin remuneración. Durante este tiempo, llamó la atención de David Hilbert, uno de los matemáticos más importantes del mundo de la época, quien la invitó a G\"{o}ttingen. Sin embargo, a las mujeres se les prohibía obtener cátedras, y solo pudo dar conferencias bajo el nombre de Hilbert, nuevamente sin remuneración. Durante este tiempo demostró lo que ahora se conoce como el teorema de Noether, que todavía se utiliza en la física teórica actual. A Noether finalmente se le concedió el derecho a enseñar en 1919. La respuesta de Hilbert a la continua resistencia de sus colegas universitarios fue, según se informa: "Caballeros, el senado de la facultad no es una casa de baños".

A finales de la década de 1920, se concentró en el trabajo en álgebra abstracta, y sus contribuciones revolucionaron el campo. En sus demostraciones, a menudo utilizaba la llamada condición de cadena ascendente, que establece que no hay una cadena estrictamente creciente infinita de ciertos conjuntos. Por ejemplo, ciertas estructuras algebraicas ahora conocidas como anillos noetherianos tienen la propiedad de que no hay secuencias infinitas de ideales $I_1 \subsetneq I_2 \subsetneq \dots$. La condición se puede generalizar a cualquier orden parcial (en álgebra, se refiere al caso especial de ideales ordenados por la relación de subconjunto), y también podemos considerar la condición de cadena descendente dual, donde cada secuencia estrictamente \emph{de}creciente en un orden parcial eventualmente termina. Si un orden parcial satisface la condición de cadena descendente, es posible utilizar la inducción a lo largo de este orden de manera similar a como podemos utilizar la inducción a lo largo del orden $<$ en $\Nat$. Tales órdenes se llaman \emph{bien fundados} o \emph{noetherianos}, y el principio de prueba correspondiente \emph{inducción noetheriana}.

Noether era judía, y cuando los nazis llegaron al poder en 1933, fue despedida de su puesto. Afortunadamente, Noether pudo emigrar a los Estados Unidos para un puesto temporal en Bryn Mawr, Pensilvania. Durante su tiempo allí, también dio conferencias en Princeton, aunque encontró que la universidad era poco acogedora para las mujeres \citep[81]{Dick1981}. En 1935, Noether se sometió a una operación para extirpar un tumor uterino. Murió de una infección como resultado de la cirugía, y fue enterrada en Bryn Mawr.

\begin{reading}
Para una biografía de Noether, véase \citet{Dick1981}. El Instituto Perimeter de Física Teórica tiene sus conferencias sobre la vida e influencia de Noether disponibles en línea \citep{Perimeter2015}. Si estás cansado de leer, \emph{Stuff You Missed in History Class} tiene un podcast sobre la vida e influencia de Noether \citep{Frey2015}. Las obras completas de Noether están disponibles en el alemán original \citep{Noether1983}.
\end{reading}

\end{document}
