% Part: history
% Chapter: biographies
% Section: ernst-zermelo

\documentclass[../../../include/open-logic-section]{subfiles}

\begin{document}

\olfileid{his}{bio}{zer}

\olsection{Ernst Zermelo}

\olphoto{zermelo-ernst}{Ernst Zermelo}

Ernst Zermelo nació el 27 de julio de 1871 en Berlín, Alemania. Tenía cinco hermanas, aunque su familia sufría de mala salud y solo tres sobrevivieron hasta la edad adulta. Sus padres también fallecieron cuando él era joven, dejándolo a él y a sus hermanos huérfanos cuando tenía diecisiete años. Zermelo tenía un profundo interés en las artes, y especialmente en la poesía. Era conocido por ser agudo, ingenioso y crítico. Sus logros matemáticos más célebres incluyen la introducción del axioma de elección (en 1904) y su axiomatización de la teoría de conjuntos (en 1908).

Los intereses de Zermelo en la universidad eran variados. Tomó cursos de física, matemáticas y filosofía. Bajo la supervisión de Hermann Schwarz, Zermelo completó su disertación \emph{Investigaciones en el Cálculo de Variaciones} en 1894 en la Universidad de Berlín. En 1897, decidió realizar más estudios en la Universidad de G\"{o}ttingen, donde fue fuertemente influenciado por el trabajo fundacional de David Hilbert. En 1899, se hizo elegible para una cátedra, pero no obtuvo una hasta once años después, posiblemente debido a su comportamiento extraño y "prisa nerviosa".

Zermelo finalmente recibió una cátedra remunerada en la Universidad de Zúrich en 1910, pero se vio obligado a retirarse en 1916 debido a la tuberculosis. Después de su recuperación, se le otorgó una cátedra honoraria en la Universidad de Friburgo en 1921. Durante este tiempo, trabajó en matemáticas fundacionales. Se irritó con los trabajos de Thoralf Skolem y Kurt G\"{o}del, y criticó públicamente sus enfoques en sus artículos. Fue destituido de su puesto en Friburgo en 1935, debido a su impopularidad y su oposición al ascenso de Hitler al poder en Alemania.

Los últimos años de la vida de Zermelo estuvieron marcados por el aislamiento. Después de su destitución en 1935, abandonó las matemáticas. Se mudó al campo donde vivió modestamente. Se casó en 1944 y se volvió completamente dependiente de su esposa, ya que se estaba quedando ciego. Zermelo perdió la vista por completo en 1951. Falleció en G\"{u}nterstal, Alemania, el 21 de mayo de 1953.

\begin{reading}
Para una biografía completa de Zermelo, véase \citet{Ebbinghaus2015}. Los artículos seminales de Zermelo de 1904 y 1908 están disponibles para leer en el alemán original \citep{Zermelo1904,Zermelo1908}. Las obras completas de Zermelo, incluidos sus escritos sobre física, están disponibles en traducción al inglés en \citep{Ebbinghaus2010,Ebbinghaus2013}.
\end{reading}

\end{document}
