% Part: history 
% Chapter: biographies 
% Section: julia-robinson
\documentclass[../../../include/open-logic-section]{subfiles}

\begin{document}

\olfileid{his}{bio}{rob}

\olsection{Julia Robinson}

\olphoto{robinson-julia}{Julia Robinson}

Julia Bowman Robinson fue una matemática estadounidense. Es conocida principalmente por su trabajo en problemas de decisión, y más famosamente por sus contribuciones a la solución del décimo problema de Hilbert. Robinson nació en St. Louis, Missouri, el 8 de diciembre de 1919. Robinson recuerda estar intrigada por los números desde niña \citep[4]{Reid1986}. A los nueve años contrajo fiebre escarlata y sufrió varios episodios recurrentes de fiebre reumática. Esto la obligó a pasar gran parte de su tiempo en cama, retrasando su educación. Aunque pudo ponerse al día con la ayuda de tutores privados, los efectos físicos de su enfermedad tuvieron un impacto duradero en su vida.

A pesar de sus luchas infantiles, Robinson se graduó de la escuela secundaria con varios premios en matemáticas y ciencias. Comenzó su carrera universitaria en San Diego State College, y se transfirió a la Universidad de California, Berkeley, como estudiante de último año. Allí fue influenciada por el matemático Raphael Robinson. Se hicieron buenos amigos y se casaron en 1941. Como esposa de un miembro de la facultad, a Robinson se le prohibió enseñar en el departamento de matemáticas de Berkeley. Aunque continuó asistiendo como oyente a clases de matemáticas, esperaba dejar la universidad y formar una familia. Poco después de su boda, sin embargo, Robinson contrajo neumonía. Le dijeron que había una acumulación sustancial de tejido cicatricial en su corazón debido a la fiebre reumática que sufrió de niña. Debido a la gravedad del tejido cicatricial, el médico predijo que no viviría más allá de los cuarenta años y le aconsejó no tener hijos \citep[13]{Reid1986}.

Robinson estuvo deprimida durante mucho tiempo, pero finalmente decidió continuar estudiando matemáticas. Regresó a Berkeley y completó su doctorado en 1948 bajo la supervisión de Alfred Tarski. Tarski había demostrado que la teoría de primer orden de los números reales era decidible, y del trabajo de G\"odel se deducía que la teoría de primer orden de los números naturales es indecidible. Era un problema abierto importante si la teoría de primer orden de los racionales es decidible o no. En su tesis \citeyearpar{Robinson1949}, Robinson demostró que no lo era.

Interesada en los problemas de decisión, Robinson intentó luego encontrar una solución al décimo problema de Hilbert. Este problema era uno de una famosa lista de 23 problemas matemáticos planteados por David Hilbert en 1900. El décimo problema pregunta si existe un algoritmo que responda, en un tiempo finito, si una ecuación polinómica con coeficientes enteros, como $3x^2 - 2y + 3 = 0$, tiene una solución en los enteros. Tales preguntas se conocen como \emph{problemas diofánticos}. Después de algunos éxitos iniciales, Robinson unió fuerzas con Martin Davis e Hilary Putnam, quienes también estaban trabajando en el problema. Lograron demostrar que los problemas diofánticos exponenciales (donde las incógnitas también pueden aparecer como exponentes) son indecidibles, y demostraron que una cierta conjetura (más tarde llamada "J.R.") implica que el décimo problema de Hilbert es indecidible \citep{DavisPutnamRobinson1961}. Robinson continuó trabajando en el problema a lo largo de la década de 1960. En 1970, el joven matemático ruso Yuri Matijasevich finalmente demostró la hipótesis J.R. El resultado combinado ahora se llama el teorema de Matijasevich-Robinson-Davis-Putnam, o teorema MRDP para abreviar. Matijasevich y Robinson se hicieron amigos y colaboraron en varios artículos. En una carta a Matijasevich, Robinson escribió una vez que "en realidad estoy muy contenta de que trabajando juntos (a miles de kilómetros de distancia) obviamente estamos haciendo más progresos de los que cualquiera de nosotros podría hacer solo" \citep[45]{Matijasevich1992}.

Robinson fue la primera mujer presidenta de la Sociedad Matemática Estadounidense, y la primera mujer en ser elegida miembro de la Academia Nacional de Ciencias. Murió el 30 de julio de 1985 a la edad de 65 años después de ser diagnosticada con leucemia.

\begin{reading}
Los artículos matemáticos de Robinson están disponibles en sus \textit{Obras Completas} \citep{Robinson1996}, que también incluye una reimpresión de su memoria biográfica de la Academia Nacional de Ciencias \citep{Feferman1994}. La hermana mayor de Robinson, Constance Reid, publicó una "Autobiografía de Julia", basada en entrevistas \citep{Reid1986}, así como una memoria completa \citep{Reid1996}. George Csicsery dirigió un breve documental sobre Robinson y el décimo problema de Hilbert \citep{Csicsery2016}. Para una breve memoria sobre las colaboraciones de Yuri Matijasevich con Robinson, y su influencia en su trabajo, véase \citep{Matijasevich1992}.
\end{reading}

\end{document}
