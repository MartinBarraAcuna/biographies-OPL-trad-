% Part: history
% Chapter: biographies 
% Section: alan-turing 

\documentclass[../../../include/open-logic-section]{subfiles}

\begin{document}

\olfileid{his}{bio}{tur} 

\olsection{Alan Turing}

Alan Turing nació en Maida Vale, Londres, el 23 de junio de 1912. Es considerado el padre de la informática teórica. El interés de Turing por las ciencias físicas y las matemáticas comenzó a una edad temprana. Sin embargo, cuando era niño, sus intereses no estaban bien representados en sus escuelas, donde se hacía hincapié en la literatura y los clásicos. En consecuencia, le fue mal en la escuela y fue reprendido por muchos de sus profesores.

\olphoto{turing-alan}{Alan Turing}

Turing asistió al King's College, Cambridge, como estudiante universitario, donde estudió matemáticas. En 1936, Turing desarrolló (lo que ahora se llama) la máquina de Turing como un intento de definir con precisión la noción de una función computable y de demostrar la indecidibilidad del problema de decisión. Fue superado en el resultado por Alonzo Church, quien demostró el resultado a través de su propio cálculo lambda. El artículo de Turing fue publicado igualmente con referencia al resultado de Church. Church invitó a Turing a Princeton, donde pasó de 1936 a 1938, y obtuvo un doctorado bajo la dirección de Church.

A pesar de su interés por la lógica, los intereses anteriores de Turing en las ciencias físicas siguieron prevaleciendo. Sus habilidades prácticas se pusieron a trabajar durante su servicio en el departamento de criptoanálisis británico en Bletchley Park durante la Segunda Guerra Mundial. Turing fue una figura central en el descifrado del código utilizado por las comunicaciones navales alemanas: el código Enigma. La experiencia de Turing en estadística y criptografía, junto con la introducción de maquinaria electrónica, le dio al equipo la capacidad de descifrar el código creando una máquina de descifrado llamada "bombe". Sus ideas también ayudaron en la creación de la primera computadora electrónica programable del mundo, el Colossus, también utilizada en Bletchley Park para descifrar el código alemán Lorenz.

Turing era gay. Sin embargo, en 1942 le propuso matrimonio a Joan Clarke, una de sus compañeras de equipo en Bletchley Park, pero luego rompió el compromiso y le confesó que era homosexual. Tuvo varios amantes a lo largo de su vida, aunque los actos homosexuales eran entonces delitos penales en el Reino Unido. En 1952, la casa de Turing fue robada por un amigo de su amante en ese momento, y al presentar una denuncia policial, Turing admitió tener una relación homosexual, bajo la impresión de que el gobierno estaba a punto de legalizar los actos homosexuales. Esto no era cierto, y fue acusado de indecencia grave. En lugar de ir a prisión, Turing optó por un tratamiento hormonal que redujo la libido. Turing fue encontrado muerto el 8 de junio de 1954, por una sobredosis de cianuro, muy probablemente un suicidio. Recibió un indulto real de la Reina Isabel II en 2013.

\begin{reading}
Para una biografía completa de Alan Turing, véase \citet{Hodges2014}. La vida y obra de Turing inspiraron una obra de teatro, \emph{Breaking the Code}, que fue producida en 1996 para televisión protagonizada por Derek Jacobi como Turing. \emph{The Imitation Game}, una película nominada al Premio de la Academia protagonizada por Benedict Cumberbatch y Keira Knightley, también se basa libremente en la vida y la época de Alan Turing en Bletchley Park \citep{Imitation2014}.

\citet{Radiolab2012} tiene varios podcasts sobre la vida y obra de Turing. El documental de BBC Horizon \emph{The Strange Life and Death of Dr. Turing} está disponible para ver en línea \citep{Sykes1992}. \citep{Theelen2012} es un breve video de una máquina de Turing LEGO en funcionamiento, hecha para honrar el centenario de Turing en 2012.

El artículo original de Turing sobre las máquinas de Turing y el problema de decisión es \citet{Turing1937}.
\end{reading}

\end{document}
