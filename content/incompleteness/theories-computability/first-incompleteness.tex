% Part: incompleteness
% Chapter: theories-computability
% Section: first-incompleteness

\documentclass[../../../include/open-logic-section]{subfiles}

\begin{document}

\olfileid{inc}{tcp}{inc}

\olsection{$\Th{Q}$ has no Complete, Consistent, !!^{axiomatizable}
  Extensions}

\begin{thm}
\ollabel{thm:first-incompleteness}
There is no complete, consistent, !!{axiomatizable} extension of
$\Th{Q}$.
\end{thm}

\begin{proof}
We already know that there is no consistent, decidable extension of
$\Th{Q}$. But if $\Th{T}$ is complete and !!{axiomatized}, then
it is decidable. 
\end{proof}

\begin{explain}
This theorems is not that far from G\"odel's original 1931 formulation
of the First Incompleteness Theorem. Aside from the more modern
terminology, the key differences are this: G\"odel has
``$\omega$-consistent'' instead of ``consistent''; and he could not
say ``!!{axiomatizable}'' in full generality, since the formal
notion of computability was not in place yet. (The formal models of
computability were developed over the following decade, including by
G\"odel, and in large part to be able to characterize the kinds of
theories that are susceptible to the G\"odel phenomenon.)

The theorem says you can't have it all, namely, completeness,
consistency, and !!{axiomatizability}. If you give up any one of
these, though, you can have the other two: $\Th{Q}$ is consistent and
computably axiomatized, but not complete; the inconsistent theory is
complete, and computably axiomatized (say, by $\{ \eq/[0][0] \}$), but
not consistent; and the set of true sentence of arithmetic is complete
and consistent, but it is not computably axiomatized.
\end{explain}
\end{document}
