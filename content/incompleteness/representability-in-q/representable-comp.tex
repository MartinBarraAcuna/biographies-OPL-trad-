% Part: incompleteness
% Chapter: representability-in-q
% Section: representable-comp

\documentclass[../../../include/open-logic-section]{subfiles}

\begin{document}

\olfileid{inc}{req}{rpc}
\olsection{Functions Representable in $\Th{Q}$ are Computable}

We'll prove that every function that is representable in~$\Th{Q}$ is
computable. We first have to establish a lemma about functions
representable in~$\Th{Q}$.

\begin{lem}\ollabel{lem:rep-q}
  If $f(x_0,
  \dots, x_k)$ is representable in~$\Th{Q}$, there is
  !!a{formula}~$!A(x_0, \dots, x_k, y)$ such that
  \[
  \Th{Q} \Proves !A_f(\num{n_0}, \dots, \num{n_k}, \num{m})
  \quad\text{iff}\quad m = f(n_0, \dots, n_k).
  \]
\end{lem}

\begin{proof}
  The ``if'' part is
\olref[int]{defn:representable-fn}\olref[int]{defn:rep:a}. The ``only
if'' part is seen as follows: Suppose $\Th{Q} \Proves !A_f(\num{n_0},
\dots, \num{n_k}, \num{m})$ but $m \neq f(n_0, \dots, n_k)$. Let $l =
f(n_0, \dots, n_k)$. By
\olref[int]{defn:representable-fn}\olref[int]{defn:rep:a}, $\Th{Q}
\Proves !A_f(\num{n_0}, \dots, \num{n_k}, \num{l})$. By
\olref[int]{defn:representable-fn}\olref[int]{defn:rep:b},
$\lforall[y][(!A_f(\num{n_0}, \dots, \num{n_k}, y) \lif \num{l} =
y)]$. Using logic and the assumption that $\Th{Q} \Proves
!A_f(\num{n_0}, \dots, \num{n_k}, \num{m})$, we get that $\Th{Q}
\Proves \eq[\num{l}][\num{m}]$. On the other hand, by
\olref[bre]{lem:q-proves-neq}, $\Th{Q} \Proves
\eq/[\num{l}][\num{m}]$. So $\Th{Q}$ is inconsistent. But that is
impossible, since $\Th{Q}$ is satisfied by the standard model (see
\olref[int][def]{def:standard-model}), $\Sat{N}{\Th{Q}}$, and
satisfiable theories are always consistent by the Soundness Theorem
(\tagrefs{prfAX/{fol:axd:sou:cor:consistency-soundness},
prfSC/{fol:seq:sou:cor:consistency-soundness},
prfND/{fol:ntd:sou:cor:consistency-soundness},
prfTab/{fol:tab:sou:cor:consistency-soundness}}).
\end{proof}

\begin{lem}
Every function that is representable in $\Th{Q}$ is computable.
\end{lem}

\begin{proof}
Let's first give the intuitive idea for why this is true. To
compute~$f$, we do the following.  List all the possible
!!{derivation}s~$\delta$ in the language of arithmetic. This is
possible to do mechanically. For each one, check if it is
!!a{derivation} of !!a{formula} of the form~$!A_f(\num{n_0}, \dots,
\num{n_k}, \num{m})$ (the !!{formula} representing $f$ in~$\Th{Q}$
from \olref{lem:rep-q}). If it is, $m = f(n_0, \dots, n_k)$ by
\olref{lem:rep-q}, and we've found the value of~$f$. The search
terminates because $\Th{Q} \Proves !A_f(\num{n_0}, \dots, \num{n_k},
\num{f(n_0, \dots, n_k)})$, so eventually we find a $\delta$ of the
right sort.

This is not quite precise because our procedure operates on
!!{derivation}s and !!{formula}s instead of just on numbers, and we
haven't explained exactly why ``listing all possible !!{derivation}s''
is mechanically possible.  But as we've seen, it is possible to code
terms, !!{formula}s, and !!{derivation}s by G\"odel numbers. We've
also introduced a precise model of computation, the general recursive
functions. And we've seen that the relation $\Prf[\Th{Q}](d,y)$, which
holds iff $d$ is the G\"odel number of !!a{derivation} of the !!{formula}
with G\"odel number~$y$ from the axioms of~$\Th{Q}$, is (primitive)
recursive. Other primitive recursive functions we'll need are
$\fn{num}$ (\olref[art][trm]{prop:num-primrec}) and $\fn{Subst}$
(\olref[art][sub]{prop:subst-primrec}).  From these, it is possible to
define~$f$ by minimization; thus, $f$ is recursive.

First, define
\begin{multline*}
  A(n_0, \dots, n_k, m) = \\
  \fn{Subst}(\fn{Subst}(\dots\fn{Subst}(\Gn{!A_f}, \fn{num}(n_0), \Gn{x_0}),\\ \dots),
  \fn{num}(n_k),  \Gn{x_k}), \fn{num}(m), \Gn{y})
\end{multline*}
This looks complicated, but it's just the function $A(n_0, \dots, n_k,
m) = \Gn{!A_f(\num{n_0}, \dots, \num{n_k}, \num{m})}$.

Now, consider the relation~$R(n_0, \dots, n_k, s)$ which holds if
$(s)_0$ is the G\"odel number of a derivation from~$\Th{Q}$ of
$!A_f(\num{n_0}, \dots, \num{n_k}, \num{(s)_1})$:
\[
R(n_0, \dots, n_k, s) \quad\text{iff}\quad \Prf[\Th{Q}]((s)_0, A(n_0,
\dots, n_k, (s)_1))
\]
If we can find an~$s$ such that $R(n_0, \dots, n_k, s)$ hold, we have
found a pair of numbers---$(s)_0$ and~$(s_1)$---such that $(s)_0$ is
the G\"odel number of !!a{derivation} of~$A_f(\num{n_0}, \dots,
\num{n_k}, (s)_1)$. So looking for $s$ is like looking for the pair
$d$ and $m$ in the informal proof.  And a computable function that
``looks for'' such an $s$ can be defined by regular minimization.
Note that $R$ is regular: for every $n_0$, \dots, $n_k$, there is
!!a{derivation}~$\delta$ of $\Th{Q} \Proves !A_f(\num{n_0}, \dots,
\num{n_k}, \num{f(n_0, \dots, n_k)})$, so $R(n_0, \dots, n_k, s)$
holds for $s = \tuple{\Gn{\delta}, f(n_0, \dots, n_k)}$.  So, we can
write $f$ as
\[
f(n_0,\dots,n_{k}) = (\umin{s}{R(n_0, \dots, n_k, s)})_1.
\]
\end{proof}

\end{document}
