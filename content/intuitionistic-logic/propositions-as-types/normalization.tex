% part: intuitionistic-logic
% chapter: propositions-as-types
% section: normalization

\documentclass[../../../include/open-logic-section]{subfiles}

\begin{document}

\olfileid{int}{pty}{nor}

\olsection{Normalization}

In this section we prove that, via some reduction order, any deduction
can be reduced to a normal deduction, which is called the
\emph{normalization property}. We will make use of the
propositions-as-types correspondence: we show that every proof term
can be reduced to a normal form; normalization for natural deduction
!!{derivation}s then follows.

Firstly we define some functions that measure the complexity of terms.
The \emph{length}~$\len{!A}$ of !!a{formula}s is defined by
\begin{align*}
  \len{p} &= 0\\
  \len{!A \land !B} &= \len{!A} + \len{!B} + 1\\
  \len{!A \lor !B} &= \len{!A} + \len{!B} + 1 \\
  \len{!A \lif !B} &= \len{!A} + \len{!B} + 1.
\end{align*}
The complexity of a redex~$M$ is measured by its \emph{cut
  rank}~$\cutrank{M}$:
\begin{align*}
  \cutrank{(\lambd[\typeof{x}{!A}][\typeof{N}{!B}])Q} &=
  \len{!A} + \len{!B} + 1 \\
  \cutrank{\ande{i}{\andi{\typeof{M}{!A}}{\typeof{N}{!B}}}} &=
  \len{!A} + \len{!B} + 1 \\
  \cutrank{\ore{\ori{i}{!A_{i}}{\typeof{M}{!A_i}}}
    {\typeof{x_1}{!A_1}}{\typeof{N_1}{!C}}
    {\typeof{x_2}{!A_2}}{\typeof{N_2}{!C}}} &=
  \len{!A} + \len{!B} + 1
\end{align*}
The complexity of a proof term is measured by the most complex
redex in it, and $0$ if it is normal:
\[
  \maxrank{M} = \max \{\cutrank{N} | N \text{ is a sub term of } M \text{
    and is redex}\}
\]

\begin{lem}\ollabel{lem:subst}
  If $\Subst{M}{\typeof{N}{!A}}{\typeof{x}{!A}}$ is a redex and $M
  \not\ident x$, then one of the following cases holds:
  \begin{enumerate}
  \item $M$ is itself a redex, or
  \item $M$ is of the form $\ande{i}{x}$, and $N$ is of the form
    $\andi{P_1}{P_2}$
  \item $M$ is of the form $\ore{i}{x_1}{P_1}{x_2}{P_2}$, and $N$ is
    of the form $\ori{i}{}{Q}$
  \item $M$ is of the form $x Q$, and $N$ is of the form
    $\lambd[x][P]$
  \end{enumerate}
  In the first case, $\cutrank{\Subst{M}{N}{x}} = \cutrank{M}$; in the
  other cases, $\cutrank{\Subst{M}{N}{x}} = \len{!A})$.
\end{lem}
\begin{proof}
  Proof by induction on~$M$.
  \begin{enumerate}
  \item If $M$ is a single variable $y$ and $y \not\ident x$, then
    $\Subst{y}{N}{x}$ is $y$, hence not a redex.
  \item If $M$ is of the form $\andi{N_1}{N_2}$, or $\lambd[x][N]$, or
    $\ori{i}{!A}{N}$, then $\Subst{M}{\typeof{N}{!A}}{\typeof{x}{!A}}$ is
    also of that form, and so is not a redex.
  \item If $M$ is of the form $\ande{i}{P}$, we consider two cases.
    \begin{enumerate}
      \item If $P$ is of the form $\andi{P_1}{P_2}$, then $M \ident
        \ande{i}{\andi{P_1}{P_2}}$ is a redex, and clearly
        \[
        \Subst{M}{N}{x} \ident
        \ande{i}{\andi{\Subst{P_1}{N}{x}}{\Subst{P_2}{N}{x}}}
        \]
        is also a redex. The cut ranks are equal.
      \item If $P$ is a single variable, it must be $x$ to make the
        substitution a redex, and $N$ must be of the form
        $\andi{P_1}{P_2}$. Now consider
        \[
        \Subst{M}{N}{x} \ident \Subst{\ande{i}{x}}{\andi{P_1}{P_2}}{x},
        \]
        which is $\ande{i}{\andi{P_1}{P_2}}$. Its cut rank is equal to
        $\cutrank{x}$, which is $\len{!A}$.
    \end{enumerate}
  \end{enumerate}
  The cases of $\ore{N}{x_1}{N_1}{x_2}{N_2}$ and $P Q$ are similar.
\end{proof}

\begin{lem}
  If $M$ contracts to $M'$, and $\cutrank{M} > \cutrank{N}$ for all
  proper redex sub-terms $N$ of $M$, then $\cutrank{M} >
  \maxrank{M'}$.
\end{lem}

\begin{proof}
  Proof by cases.
  \begin{enumerate}
  \item If $M$ is of the form $\ande{i}{\andi{M_1}{M_2}}$, then $M'$ is
    $M_i$; since any sub-term of $M_i$ is also proper
    sub-term of $M$, the claim holds.
  \item If $M$ is of the form
    $(\lambd[\typeof{x}{!A}][N])\typeof{Q}{!A}$, then $M'$ is
    $\Subst{N}{\typeof{Q}{!A}}{\typeof{x}{!A}}$.  Consider a redex in
    $M'$.  Either there is corresponding redex in $N$ with equal cut
    rank, which is less than $\cutrank{M}$ by assumption, or the cut
    rank equals $\len{!A}$, which by definition is less than
    $\cutrank{(\lambd[x^{!A}][N])Q}$.
  \item If $M$ is of the form
    \[
    \ore{\ori{i}{}{\typeof{N}{!A_i}}}
        {\typeof{x_1}{!A_1}}{\typeof{N_1}{!C}}
        {\typeof{x_2}{!A_2}}{\typeof{N_2}{!C}},
    \]
    then $M' \ident \Subst{N_i}{N}{\typeof{x_i}{!A_i}}$.  Consider a
    redex in~$M'$. Either there is corresponding redex in $N_i$ with
    equal cut rank, which is less than $\cutrank{M}$ by assumption; or
    the cut rank equals $\len{!A_i}$, which by definition is less than
    $\cutrank{\ore{\ori{i}{}{\typeof{N}{!A_i}}}
      {\typeof{x_1}{!A_1}}{\typeof{N_1}{!C}}
      {\typeof{x_2}{!A_2}}{\typeof{N_2}{!C}}}$.
  \end{enumerate}
\end{proof}

\begin{thm}
  All proof terms reduce to normal form; all !!{derivation}s reduce to
  normal !!{derivation}s.
\end{thm}

\begin{proof}
  The second follows from the first. We prove the first by complete
  induction on $m=\maxrank{M}$, where $M$ is a proof term.
  \begin{enumerate}

  \item If $m=0$, $M$ is already normal.
  \item
    Otherwise, we proceed by induction on~$n$, the number of
    redexes in $M$ with cut rank equal to~$m$.
    \begin{enumerate}
    \item If $n=1$, select any redex $N$ such that $m = \cutrank{N} >
      \cutrank{P}$ for any proper sub-term $P$ which is also a redex
      of course. Such a redex must exist, since any term only has
      finitely many subterms.

      Let $N'$ denote the reductum of $N$. Now by the lemma
      $\maxrank{N'} < \maxrank{N}$, thus we can see that $n$, the
      number of redexes with $\cutrank=m$ is decreased. So $m$ is
      decreased (by $1$ or more), and we can apply the inductive
      hypothesis for~$m$.
    \item For the induction step, assume $n>1$. the process is
      similar, except that $n$ is only decreased to a positive number
      and thus $m$ does not change. We simply apply the induction
      hypothesis for~$n$.
    \end{enumerate}
\end{enumerate}
\end{proof}

The normalization of terms is actually not specific to the reduction
order we chose. In fact, one can prove that regardless of the order in
which redexes are reduced, the term always reduces to a normal
form. This property is called \emph{strong normalization}.
\end{document}
