% part: intuitionistic-logic
% chapter: propositions-as-types
% section: reduction

\documentclass[../../../include/open-logic-section]{subfiles}

\begin{document}

\olfileid{int}{pty}{red}

\olsection{Reduction}

In natural deduction !!{derivation}s, an introduction rule that is
followed by an elimination rule is redundant. For instance, the
!!{derivation}
\begin{gather*}
  \AxiomC{$!A$}
  \AxiomC{$!A \lif !B$}
  \RightLabel{$\Elim{\lif}$}
  \BinaryInfC{$!B$}
  \AxiomC{$\Discharge{!C}{}$}
  \RightLabel{$\Intro{\land}$}
  \BinaryInfC{$!B \land !C$}
  \RightLabel{$\Elim{\land}$}
  \UnaryInfC{$!B$}
  \RightLabel{$\Intro{\lif}$}
  \UnaryInfC{$!C \lif !B$}
  \DisplayProof
\end{gather*}
can be replaced with the simpler !!{derivation}:
\begin{gather*}
  \AxiomC{$!A$}
  \AxiomC{$!A \lif !B$}
  \RightLabel{$\Elim{\lif}$}
  \BinaryInfC{$!B$}
  \RightLabel{$\Intro{\lif}$}
  \UnaryInfC{$!C \lif !B$}
  \DisplayProof
\end{gather*}

As we see, an $\Intro{\land}$ followed by $\Elim{\land}$ ``cancel
out.'' In general, we see that the conclusion of $\Elim{\land}$ is
always the formula on one side of the conjunction, and the premises of
$\Intro{\land}$ requires both sides of the conjunction, thus if we
need a derivation of either side, we can simply use that derivation
without introducing the conjunction followed by eliminating it.

Thus in general we have
\begin{gather*}
  \bottomAlignProof
  \AxiomC{}
  \RightLabel{$D_1$}
  \DeduceC{$!A_1$}
  \AxiomC{}
  \RightLabel{$D_2$}
  \DeduceC{$!A_2$}
  \RightLabel{$\Intro{\land}$}
  \BinaryInfC{$!A_1 \land !A_2$}
  \RightLabel{$\Elim{\land}_i$}
  \UnaryInfC{$!A_i$}
  \DisplayProof\bottomAlignProof
  \quad
  \redone
  \quad
  \AxiomC{}
  \RightLabel{$D_i$}
  \DeduceC{$!A_i$}
  \DisplayProof
\end{gather*}

The $\redone$ symbol has a similar meaning as in the lambda calculus,
i.e., a single step of a reduction. In the proof term syntax for
!!{derivation}s, the above reduction rule thus becomes:
\begin{gather*}
  (\Gamma, p_i \tuple{\typeof{M_1}{!A_1}, \typeof{M_2}{!A_2}}) \redone
  (\Gamma, M_i)
\end{gather*}
In the typed lambda calculus, this is the beta reduction rule for
the product type.

Note the type annotation on $M_1$ and $M_2$: while in the standard
term syntax only $\lambd[\typeof{x}{!A}][N]$ has such notion, we reuse
the notation here to remind us of the formula the term is associated
with in the corresponding natural deduction !!{derivation}, to reveal
the correspondence between the two kinds of syntax.

In natural deduction, a pair of inferences such as those on the left,
i.e., a pair that is subject to cancelling is called a \emph{cut}. In
the typed lambda calculus the term on the left of $\redone$ is called
a \emph{redex}, and the term to the right is called the
\emph{reductum}. Unlike untyped lambda calculus, where only
$(\lambd[x][N])Q$ is considered to be redex, in the typed lambda
calculus the syntax is extended to terms involving $\andi{N}{M}$,
$\ande{i}{N}$, $\ori{i}{!A}{N}$, $\ore{N}{x_1}{M_1}{x_2}{M_2}$, and
$\falsee{N}$, with corresponding redexes.

Similarly we have reduction for disjunction:
\begin{gather*}
  \bottomAlignProof
  \AxiomC{}
  \RightLabel{$D$}
  \DeduceC{$!A_i$}
  \RightLabel{$\Intro{\lor}$}
  \UnaryInfC{$!A_1 \lor !A_2$}
  \AxiomC{$\Discharge{!A_1}{u}$}
  \RightLabel{$D_1$}
  \DeduceC{$!C$}
  \AxiomC{$\Discharge{!A_2}{u}$}
  \RightLabel{$D_2$}
  \DeduceC{$!C$}
  \DischargeRule{$\Elim{\lor}$}{u}
  \TrinaryInfC{$!C$}
  \DisplayProof\bottomAlignProof
  \quad
  \redone
  \quad
  \AxiomC{}
  \RightLabel{$D$}
  \DeduceC{$!A_i$}
  \RightLabel{$D_i$}
  \DeduceC{$!C$}
  \DisplayProof
\end{gather*}
This corresponds to a reduction on proof terms:
\begin{gather*}
  (\Gamma, \ore{\ori{i}{!A_i}{\typeof{M}{!A_i}}}{\typeof{x_1}{!A_1}}{\typeof{N_1}{!C}}
  {\typeof{x_2}{!A_2}}{\typeof{N_2}{!C}}) \redone (\Gamma, \Subst{\typeof{N_i}{!C}}{\typeof{M}{!A_i}}{\typeof{x_i}{!A_i}})
\end{gather*}
This is the beta reduction rule of for sum types.  Here,
$\Subst{M}{N}{x}$ means replacing all assumptions denoted by variable
$x$ in $M$ with $N$,

It would be nice if we pass the context $\Gamma$ to the substitution
function so that it can check if the substitution makes sense.  For
example, $\Subst{x y}{a b}{y}$ does not make sense under the context
$\{x:!A \lif !D, y:!A, a:!B\lif !C, b:!B\}$ since then we would be
substituting !!a{derivation} of~$!C$ where !!a{derivation} of~$!A$ is
expected. However, as long as our usage of substitution is careful
enough to avoid such errors, we won't have to worry about such
conflicts. Thus we can define it recursively as we did for untyped
lambda calculus as if we are dealing with untyped terms.

Finally, the reduction of the function type corresponds to removal of
a detour of a $\Intro\lif$ followed by a $\Elim\lif$.
\begin{gather*}
  \bottomAlignProof
  \AxiomC{$\Discharge{!A}{u}$}
  \RightLabel{$D$}
  \DeduceC{$!B$}
  \DischargeRule{$\Intro{\lif}$}{u}
  \UnaryInfC{$!A \lif !B$}
  \AxiomC{}
  \RightLabel{$D'$}
  \DeduceC{$!A$}
  \RightLabel{$\Elim{\lif}$}
  \BinaryInfC{$!B$}
  \DisplayProof\bottomAlignProof
  \quad
  \redone
  \quad
  \AxiomC{}
  \RightLabel{$D'$}
  \DeduceC{$!A$}
  \RightLabel{$D$}
  \DeduceC{$!B$}
  \DisplayProof
\end{gather*}
For proof terms, this amounts to ordinary beta reduction:
\begin{gather*}
  (\Gamma, (\lambd[\typeof{x}{!A}][\typeof{N}{!B}])\typeof{Q}{!A})
  \redone (\Gamma, \Subst{\typeof{N}{!B}}{\typeof{Q}{!A}}{\typeof{x}{!A}})
\end{gather*}

Absurdity has only an elimination rule and no introduction rule, thus
there is no such reduction for it.

Note that the above notion of reduction concerns only deductions with
a cut at the end of !!a{derivation}. We would of course like to extend
it to reduction of cuts anywhere in !!a{derivation}, or reductions of
subterms of proof terms which constitute redexes. Note that, however,
the conclusion of the reduction does not change after reduction, thus
we are free to continue applying rules to both sides of $\redone$. The
resulting pairs of trees constitutes an extended notion of reduction;
it is analogous to compatibility in the untyped lambda calculus.

It's easy to see that the context $\Gamma$ does not change during the
reduction (both the original and the extended version), thus it's
unnecessary to mention the context when we are discussing
reductions. In what follows we will assume that every term is
accompanied by a context which does no change during reduction. We
then say ``proof term'' when we mean a proof term accompanied by a
context which makes it well-typed.

As in lambda calculus, the notion of normal-form term and normal
deduction is given:
\begin{defn}
  A proof term with no redex is said to be in \emph{normal form};
  likewise, !!a{derivation} without cuts is a \emph{normal
    !!{derivation}}. A proof term is in normal form if and
  only if its counterpart !!{derivation} is normal.
\end{defn}
\end{document}
