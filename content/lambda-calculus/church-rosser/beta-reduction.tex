% Part: lambda-calculus
% Chapter: church-rosser
% Section: beta-reduction

\documentclass[../../../include/open-logic-section]{subfiles}

\begin{document}

\olfileid{lam}{cr}{b}

\olsection{$\beta$-reduction}

\begin{lem}\ollabel{lem:one-par}
  If $M \bredone M'$, then $M \bredpar M'$.
\end{lem}
\begin{proof} If $M \bredone M'$, then $M$ is
    $(\lambd[x][N])Q$, $M'$ is $\Subst{N}{Q}{x}$, for some
    $x$, $N$, and~$Q$. Since $N \bredpar N$ and $Q \bredpar Q$ by
    \olref[pb]{thm:refl}, we immediately have $(\lambd[x][N])Q
    \bredpar \Subst{N}{Q}{x}$ by \olref[pb]{defn:bredpar}\olref[pb]{defn:bredpar4}.
\end{proof}

\begin{lem}\ollabel{lem:par-red}
  If $M \bredpar M'$, then $M \bred M'$.
\end{lem}

\begin{proof} By induction on the derivation of $M \bredpar M'$.
  \begin{enumerate}
  \item The last rule is \olref[pb]{defn:bredpar1}: Then $M$ and $M'$
    are just $x$, and $x \bred x$.
  \item The last rule is \olref[pb]{defn:bredpar2}: $M$ is 
    $\lambd[x][N]$ and $M'$ is $\lambd[x][N']$ for some $x$, $N$, $N'$, where
    $N \bredpar N'$. By induction hypothesis we have $N \bred N'$. Then
    $\lambd[x][N] \bred \lambd[x][N']$ (by the same series of
    $\bredone$ contractions as $N \bred N'$).
  \item The last rule is \olref[pb]{defn:bredpar3}: $M$ is 
    $PQ$ and $M'$ is $P'Q'$ for some $P$, $Q$, $P'$, $Q'$, where $P \bredpar P'$
    and $Q \bredpar Q'$. By induction hypothesis we have $P \bred P'$ and $Q \bred
    Q'$.  So $PQ \bred P'Q'$ by the reduction sequence $P \bred P'$ followed
    by the reduction $Q \bred Q'$.
  \item The last rule is \olref[pb]{defn:bredpar4}: $M$ is
    $(\lambd[x][N])Q$ and $M'$ is  $\Subst{N'}{Q'}{x}$ for some $x$,
    $N$, $M'$, $Q$, $Q'$, where $N \bredpar N'$ and $Q \bredpar Q'$.
    By induction hypothesis we get $Q \bred Q'$ and $N \bred N'$. So
    $(\lambd[x][N])Q \bred \Subst{N'}{Q'}{x}$ by $N \bred N'$ followed
    by $Q \bred Q'$ and finally contraction of $(\lambd[x][N'])Q'$ to
    $\Subst{N'}{Q'}{x}$.
  \end{enumerate}
\end{proof}

\begin{lem}\ollabel{lem:str}
  $\bred$ is the smallest transitive relation containing $\bredpar$.
\end{lem}

\begin{proof}
  Let $\xred$ be the smallest transitive relation containing
  $\bredpar$.

  $\bred \subseteq \xred$: Suppose $M \bred M'$, i.e., $M \ident M_1
  \bredone \dots \bredone M_k \ident M'$. By \olref{lem:one-par}, $M
  \ident M_1 \bredpar \dots \bredpar M_k \ident M'$. Since is $\xred$
  contains $\bredpar$ and is transitive, $M \xred M'$.

  $\xred \subseteq \bred$: Suppose $M \xred M'$, i.e., $M
  \ident M_1 \bredpar \dots \bredpar M_k \ident M'$. By \olref{lem:par-red}, $M \ident M_1
  \bred \dots \bred M_k \ident M'$. Since $\bred$ is transitive, $M
  \bred M'$.
\end{proof}

\begin{thm}\ollabel{thm:cr}
  $\bred$ satisfies the Church--Rosser property.
\end{thm}

\begin{proof}
  Immediate from \olref[dap]{thm:str}, \olref[pb]{thm:cr}, and \olref{lem:str}.
\end{proof}

\end{document}
