% Part: lambda-calculus
% Chapter: syntax
% Section: de-bruijn

\documentclass[../../../include/open-logic-section]{subfiles}

\begin{document}

\olfileid{lam}{syn}{deb}

\olsection{The De Bruijn Index}

$\alpha$-Equivalence is very natural, as terms that are
$\alpha$-equivalent ``mean the same.'' In fact, it is possible to give
a syntax for lambda terms which does not distinguish terms that can be
$\alpha$-converted to each other. The best known replaces variables by
their \emph{De Bruijn index}.

When we write $\lambd[x][M]$, we explicitly state that $x$ is the
parameter of the function, so that we can use $x$ in $M$ to refer to
this parameter. In the de Bruijn index, however, parameters have no
name and reference to them in the function body is denoted by a number
denoting the levels of abstraction between them. For example, consider
the example of $\lambd[x][\lambd[y][y x]]$: the outer abstraction is on
binds the variable~$x$; the inner abstraction binds the variable
is~$y$; the sub-term $y x$ lies in the scope of the inner abstraction:
there is no abstraction between $y$ and its abstract~$\lambd[y]$, but
one abstract between $x$ and its abstract~$\lambd[x]$. Thus we write
$0\, 1$ for $y x$, and $\lambd[][\lambd[][01]]$ for the entire term.

\begin{defn}
  De Bruijn terms are inductively defines as follows:
  \begin{enumerate}
  \item $n$, where $n$ is any natural number.
  \item $PQ$, where $P$ and $Q$ are both De Bruijn terms.
  \item $\lambd[][N]$, where $N$ is a De Bruijn term.
  \end{enumerate}
\end{defn}

A formalized translation from ordinary lambda terms to De Bruijn
indexed terms is as follows:
\begin{defn}
  \begin{align*}
    F_\Gamma(x) &= \Gamma(x) \\
    F_\Gamma(PQ) &= F_\Gamma(P)F_\Gamma(Q) \\
    F_\Gamma(\lambd[x][N]) &= \lambd[][F_{x,\Gamma}(N)]
  \end{align*}
  where $\Gamma$ is a list of variables indexed from zero, and
  $\Gamma(x)$ denotes the position of the variable $x$ in $\Gamma$.
  For example, if $\Gamma$ is $x,y,z$, then $\Gamma(x)$ is $0$ and
  $\Gamma(z)$ is $2$.
  
  $x,\Gamma$ denotes the list resulted from pushing $x$ to the head of
  $\Gamma$; for instance, continuing the $\Gamma$ in last example,
  $w,\Gamma$ is $w,x,y,z$.
\end{defn}

Recovering a standard lambda term from a de Bruijn term is done as
follows:

\begin{defn}
  \begin{align*}
    G_\Gamma(n) &= \Gamma[n] \\
    G_\Gamma(PQ) &= G_\Gamma(P) G_\Gamma(Q) \\
    G_\Gamma(\lambd[][N]) &= \lambd[x][G_{x,\Gamma}(N)]
  \end{align*}
  where $\Gamma$ is again a list of variables indexed from zero, and
  $\Gamma[n]$ denotes the variable in position~$n$. For example,
  if $\Gamma$ is $x,y,z$, then $\Gamma[1]$ is $y$.

  The variable $x$ in last equation is chosen to be any variable that
  not in $\Gamma$.
\end{defn}

Here we give some results without proving them:

\begin{prop}
  If $M \aconvone M'$, and $\Gamma$ is any list containing $\FV{M}$, then
  $F_\Gamma(M) \eqs F_\Gamma(M')$.
\end{prop}

\end{document}
