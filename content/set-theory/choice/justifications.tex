\documentclass[../../../include/open-logic-section]{subfiles}

\begin{document}

\olfileid{sth}{choice}{justifications}
\olsection{Intrinsic Considerations about Choice}

The broader question, then, is whether Well-Ordering, or Choice, or
indeed the comparability of all sets as regards their size---it
doesn't matter which---can be justified. 

Here is an attempted \emph{intrinsic} justification. Back in
\olref[z][story]{sec}, we introduced several principles
about the hierarchy. One of these is worth restating:
\begin{enumerate}
	\item[] \stagesacc. For any stage $S$, and for any sets which were
	formed \emph{before} stage $S$: a set is formed at stage $S$ whose
	members are exactly those sets. Nothing else is formed at
	stage~$S$. 
\end{enumerate}
In fact, many authors have suggested that the Axiom of Choice can be
justified via (something like) this principle. We will briefly provide
a gloss on that approach.

We will start with a simple little result, which offers \emph{yet
another} equivalent for Choice:

\begin{thm}[in $\ZF$]\ollabel{choiceset}
Choice is equivalent to the following principle. If the !!{element}s
of $A$ are disjoint and non-empty, then there is some $C$ such that $C
\cap x$ is a singleton for every $x \in A$. (We call such a $C$ a
{choice set} for $A$.)
\end{thm}

The proof of this result is straightforward, and we leave it as an
exercise for the reader. 

\begin{prob}
Prove \olref[sth][choice][justifications]{choiceset}. If you struggle,
you can find a proof in \cite[pp.~242--3]{Potter2004}.
\end{prob}

The essential point is that a choice set for $A$ is just the range of
a choice function for $A$. So, to justify Choice, we can simply try to
justify its equivalent formulation, in terms of the existence of
choice sets. And we will now try to do exactly that. 

Let $A$'s !!{element}s be disjoint and non-empty. By \stageshier{}
(see \olref[z][story]{sec}), $A$ is formed at some stage~$S$. Note
that all the !!{element}s of $\bigcup A$ are available before stage
$S$. Now, by \stagesacc{}, for \emph{any} sets which were formed
before~$S$, a set is formed whose members are exactly those sets.
Otherwise put: every \emph{possible} collections of earlier-available
sets will exist at~$S$. But it is certainly \emph{possible} to select
objects which could be formed into a choice set for~$A$; that is just
some very specific subset of $\bigcup A$. So: some such choice set
exists, as required.

Well, that's a \emph{very} quick attempt to offer a justification of
Choice on intrinsic grounds. But, to pursue this idea further, you
should read Potter's (\citeyear[\S14.8]{Potter2004}) neat development
of it.

\end{document}