\documentclass[../../../include/open-logic-section]{subfiles}

\begin{document}

\olfileid{sth}{choice}{hartogs}
\olsection{Comparability and Hartogs' Lemma}

That's the plus side. Here's the minus side. Without Choice, things
get \emph{messy}. To see why, here is a nice result due to
\cite{Hartogs1915}:

\begin{lem}[\emph{in $\ZF$}]\ollabel{HartogsLemma}
For any set $A$, there is an ordinal $\alpha$ such that $\cardnless{\alpha}{A}$
\end{lem}

\begin{proof}
If $B \subseteq A$ and $R \subseteq B^2$, then $\tuple{B, R} \subseteq
V_{\setrank{A}+4}$ by
\olref[ord-arithmetic][using-addition]{rankcomputation}. So, using
Separation, consider:
\[
	C = \Setabs{\tuple{B, R} \in V_{\setrank{A}+5}}{B\subseteq A 
	\text{ and $\tuple{B, R}$ is a well-ordering}}
\]
Using Replacement and
\olref[ordinals][ordtype]{thmOrdinalRepresentation}, form the set: 
\[
	\alpha = \Setabs{\ordtype{B, R}}{\tuple{B, R} \in C}.
\]
By \olref[ordinals][basic]{corordtransitiveord}, $\alpha$ is an
ordinal, since it is a transitive set of ordinals. After all, if
$\gamma \in \beta \in \alpha$, then $\beta = \ordtype{B, R}$ for some
$B \subseteq R$, whereupon $\gamma = \ordtype{B_b, R_b}$ for some $b
\in B$ by \olref[ordinals][iso]{wellordinitialsegment}, so that
$\gamma \in \alpha$. 

For reductio, suppose there is !!a{injection} $f \colon \alpha \to A$.
Then, where:
\begin{align*}
	B &= \ran{f}\\
	R &= \Setabs{\tuple{f(\alpha), f(\beta)} \in A \times A}{\alpha \in \beta}.
\end{align*}
Clearly $\alpha = \ordtype{B, R}$ and $\tuple{B, R} \in C$. So $\alpha
\in \alpha$, which is a contradiction.
\end{proof}

This entails a deep result:

\begin{thm}[\emph{in $\ZF$}]
The following claims are equivalent:
\begin{enumerate}
	\item\ollabel{equivwo} The Axiom of Well-Ordering
	\item\ollabel{equivcompare} Either $\cardle{A}{B}$ or
	$\cardle{B}{A}$, for any sets $A$ and $B$
\end{enumerate}
\end{thm}

\begin{proof}
\emph{\olref{equivwo} $\Rightarrow$ \olref{equivcompare}.} Fix $A$ and
$B$. Invoking \olref{equivwo}, there are well-orderings $\tuple{A, R}$
and $\tuple{B, S}$. Invoking
\olref[ordinals][ordtype]{thmOrdinalRepresentation}, let $f \colon
\alpha \to \tuple{A, R}$ and $g \colon \beta \to \tuple{B, S}$ be
isomorphisms. By \olref[sth][ordinals][basic]{ordinalsaresubsets}, either $\alpha \subseteq \beta$ or $\beta \subseteq \alpha$. If $\alpha \subseteq \beta$, then $\comp{f^{-1}}{g} \colon A \to B$ is !!a{injection}, and hence
$\cardle{A}{B}$; similarly, if $\beta \subseteq \alpha$ then $\cardle{B}{A}$.

\emph{\olref{equivcompare} $\Rightarrow$ \olref{equivwo}.} Fix $A$; by
\olref{HartogsLemma} there is some ordinal $\beta$ such that
$\cardnless{\beta}{A}$. Invoking \olref{equivcompare}, we have
$\cardle{A}{\beta}$. So there is some !!{injection} $f \colon A \to
\beta$, and we can use this injection to well-order the elements of
$A$, by defining an order $\Setabs{\tuple{a, b} \in A \times A}{f(a)
\in f(b)}$.
\end{proof}
\noindent
As an immediate consequence: if Well-Ordering fails, then some sets
are \emph{literally incomparable} with regard to their size. So, if
Well-Ordering fails, then transfinite cardinal arithmetic will be
messy. For example, we will have to abandon the idea that if $A$ and
$B$ are infinite then $\cardeq{\cardeq{A \disjointsum B}{A \times
B}}{M}$, where $M$ is the larger of $A$ and $B$ (see
\olref[card-arithmetic][simp]{cardplustimesmax}). The problem is
simple: if we cannot \emph{compare} the size of  $A$ and $B$, then it
is nonsensical to ask which is larger.

\end{document}