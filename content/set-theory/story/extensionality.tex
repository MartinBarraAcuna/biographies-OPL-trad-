\documentclass[../../../include/open-logic-section]{subfiles}

\begin{document}

\olfileid{sth}{story}{extensionality}
\olsection{Extensionality}

The very first thing to say is that sets are individuated by their
!!{element}s. More precisely:

\begin{axiom}[Extensionality]
If sets $A$ and $B$ have the same !!{element}s, then $A$ and $B$ are
the same set.
\[
  \lforall[A][\lforall[B][(\lforall[x][(x \in A \liff x \in B)] \lif
  \eq[A][B])]]
\]
\end{axiom}

We assumed this throughout \olref[sfr][][]{part}. But it bears
repeating. The Axiom of Extensionality expresses the basic idea that a
set is determined by its !!{element}s. (So sets might be contrasted with
\emph{concepts}, where precisely the same objects might fall under
many different concepts.) 

Why embrace this principle? Well, it is plausible to say that any
denial of Extensionality is a decision to abandon anything which might
even be called \emph{set theory}. Set theory is no more nor less than
the theory of extensional collections. 

The real challenge in \olref[sth][][]{part}, though, is to lay
down principles which tell us \emph{which sets exist}. And it turns
out that the only truly ``obvious'' answer to this question is
provably wrong.

\end{document}