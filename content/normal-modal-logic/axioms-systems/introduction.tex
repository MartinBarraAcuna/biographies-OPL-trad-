% Part: normal-modal-logic
% Chapter: axioms-systems
% Section: introduction

\documentclass[../../../include/open-logic-section]{subfiles}

\begin{document}

\olfileid{nml}{axs}{int}

\olsection{Introduction}

We have a semantics for the basic modal language in terms of modal
models, and a notion of !!a{formula} being valid---true at all worlds
in all models---or valid with respect to some class of models or
frames---true at all worlds in all models in the class, or based on
the frame. Logic usually connects such semantic characterizations of
validity with a proof-theoretic notion of !!{derivability}. The aim is
to define a notion of !!{derivability} in some system such that
!!a{formula} is !!{derivable} iff it is valid.

The simplest and historically oldest !!{derivation} systems are
so-called Hilbert-type or axiomatic !!{derivation} systems.
Hilbert-type !!{derivation} systems for many modal logics are
relatively easy to construct: they are simple as objects of
metatheoretical study (e.g., to prove soundness and
completeness). However, they are much harder to use to prove
!!{formula}s in than, say, natural deduction systems.

In Hilbert-type !!{derivation} systems, a derivation of !!a{formula}
is a sequence of !!{formula}s leading from certain axioms, via a
handful of inference rules, to the !!{formula} in question.  Since we
want the !!{derivation} system to match the semantics, we have to
guarantee that the set of !!{derivable} formulas are true in all
models (or true in all models in which all axioms are true). We'll
first isolate some properties of modal logics that are necessary for
this to work: the ``normal'' modal logics. For normal modal logics,
there are only two inference rules that need to be assumed: modus
ponens and necessitation.  As axioms we take all (substitution
instances) of tautologies, and, depending on the modal logic we deal
with, a number of modal axioms. Even if we are just interested in the
class of all models, we must also count all substitution instances
of~$\Ax{K}$\iftag{notprvDiamond}{}{ and~$\Ax{Dual}$} as axioms. This
alone generates the minimal normal modal logic~$\Log K$.

\begin{defn}
The rule of \emph{modus ponens} is the inference schema
\begin{prooftree}
\AxiomC{$!A$}
\AxiomC{$!A \lif !B$}
\RightLabel{\MP}
\BinaryInfC{$!B$}
\end{prooftree}
We say !!a{formula}~$!B$ \emph{follows from}~!!{formula}s $!A$, $!C$
by modus ponens iff $!C \ident !A \lif !B$.
\end{defn}

\begin{defn}
The rule of \emph{necessitation} is the inference schema
\begin{prooftree}
\AxiomC{$!A$}
\RightLabel{\Nec}
\UnaryInfC{$\Box !A$}
\end{prooftree}
We say the !!{formula}~$!B$ follows from the !!{formula}s $!A$ by
necessitation iff $!B \ident \Box !A$.
\end{defn}

\begin{defn}
A \emph{!!{derivation}} from a set of axioms~$\Sigma$ is a sequence of
!!{formula}s $!B_1$, $!B_2$, \dots, $!B_n$, where each $!B_i$ is
either
\begin{enumerate}
\item a substitution instance of a tautology, or
\item a substitution instance of !!a{formula} in~$\Sigma$, or
\item follows from two !!{formula}s $!B_j$, $!B_k$ with $j$, $k < i$
  by modus ponens, or
\item follows from !!a{formula}~$!B_j$ with $j < i$ by necessitation.
\end{enumerate}
If there is such !!a{derivation} with $!B_n \ident !A$, we say that
$!A$ is \emph{!!{derivable} from $\Sigma$}, in symbols $\Sigma \Proves !A$.
\end{defn}

With this definition, it will turn out that the set of !!{derivable}
formulas forms a normal modal logic, and that any !!{derivable}
!!{formula} is true in every model in which every axiom is true. This
property of !!{derivation}s is called \emph{soundness}. The converse,
\emph{completeness}, is harder to prove.

\end{document}
