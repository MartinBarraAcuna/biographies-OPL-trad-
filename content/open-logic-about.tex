\chapter*{Sobre el Open Logic Project}
\addcontentsline{toc}{chapter}{Sobre el Open Logic Project}


The \textit{Open Logic Text} es un libro de texto colaborativo de código abierto sobre meta-lógica formal y métodos formales, comenzando en un nivel intermedio (es decir, después de un curso introductorio de lógica formal). Aunque está dirigido a un público no matemático (en particular, estudiantes de filosofía e informática), es riguroso.

La cobertura de algunos temas actualmente incluidos puede no estar completa, y muchas secciones aún requieren una revisión sustancial. Planeamos expandir el texto para cubrir más temas en el futuro. También planeamos agregar características al texto, como un glosario, una lista de lecturas adicionales, notas históricas, imágenes, mejores explicaciones, secciones que expliquen la relevancia de los resultados para la filosofía, la informática y las matemáticas, y más problemas y ejemplos. Si encuentras un error o tienes una sugerencia, \href{https://github.com/OpenLogicProject/OpenLogic/wiki/Contributing}{por favor, informa al equipo del proyecto}.

El proyecto opera con el espíritu del código abierto. No solo el texto está disponible gratuitamente, sino que también proporcionamos el código fuente LaTeX bajo la licencia Creative Commons Attribution, que otorga a cualquiera el derecho a descargar, usar, modificar, reorganizar, convertir y redistribuir nuestro trabajo, siempre y cuando den el crédito apropiado. Por favor, consulta el sitio web del Proyecto de Lógica Abierta en \href{http://openlogicproject.org/}{openlogicproject.org} para obtener información adicional.

\chapter*{Explicación de este documento}
Este documento corresponde al apéndice de biografías del documento maestro del Open Logic Project. Decidí hacer un documento solo con este apéndice para poder darle mayor visibilidad, Además, la corta extensión lo hace ideal para ser un documento por sí mismo y enfocado al público general.
\vspace{1cm}

\noindent \href{https://sites.google.com/view/achefa/inicio}{\textbf{Página Oficial de la Asociación Chilena de Estudiantes de Filosofía Analítica (\textit{AChEFA})}}.

