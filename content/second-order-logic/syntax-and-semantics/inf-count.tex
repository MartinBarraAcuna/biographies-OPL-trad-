% Part: second-order-logic
% Chapter: syntax-and-semantics
% Section: size-of-domain

\documentclass[../../../include/open-logic-section]{subfiles}

\begin{document}

\olfileid{sol}{syn}{siz}

\olsection{Describing Infinite and \usetoken{S}{enumerable}
  \usetoken{P}{domain}}

A set~$M$ is (Dedekind) infinite iff there is !!a{injective} function~$f\colon M
\to M$ which is not !!{surjective}, i.e., with $\dom{f} \neq M$. In
first-order logic, we can consider a one-place !!{function}~$f$ and
say that the function~$\Assign{f}{M}$ assigned to it in
!!a{structure}~$\Struct{M}$ is !!{injective} and $\ran{f} \neq
\Domain{M}$:
\[
\lforall[x][\lforall[y][(\eq[f(x)][f(y)] \to \eq[x][y])]] \land
\lexists[y][\lforall[x][\eq/[y][f(x)]]].
\]
If $\Struct{M}$ satisfies this !!{sentence}, $\Assign{f}{M}:
\Domain{M} \to \Domain{M}$ is !!{injective}, and so $\Domain{M}$ must
be infinite. If $\Domain{M}$ is infinite, and hence such a function
exists, we can let $\Assign{f}{M}$ be that function and $\Struct{M}$
will satisfy the !!{sentence}.  However, this requires that our language
contains the non-logical symbol~$f$ we use for this purpose. In
second-order logic, we can simply say that such a function
\emph{exists}. This no-longer requires~$f$, and we obtain the
!!{sentence} in pure second-order logic
\[
\fn{Inf} \ident \lexists[u][(\lforall[x][\lforall[y][(\eq[u(x)][u(y)]
      \lif \eq[x][y])]] \land \lexists[y][\lforall[x][\eq/[y][u(x)]]])].
\]
$\Sat{M}{\fn{Inf}}$ iff $\Domain{M}$ is infinite.  We can then define
$\fn{Fin} \ident \lnot \fn{Inf}$; $\Sat{M}{\fn{Fin}}$ iff $\Domain{M}$
is finite. No single !!{sentence} of pure first-order logic can
express that the !!{domain} is infinite although an infinite set of
them can. There is no set of !!{sentence}s of pure first-order logic
that is satisfied in !!a{structure} iff its domain is finite.

\begin{prop}
$\Sat{M}{\fn{Inf}}$ iff $\Domain{M}$ is infinite.
\end{prop}

\begin{proof}
$\Sat{M}{\fn{Inf}}$ iff
  $\Sat{M}{\lforall[x][\lforall[y][(\eq[u(x)][u(y)] \lif \eq[x][y])]]
    \land \lexists[y][\lforall[x][\eq/[y][u(x)]]]}[s]$ for
  some~$s$. If it does, $s(u)$ is !!a{injective} function, and some $y
  \in \Domain{M}$ is not in the domain of~$s(u)$. Conversely, if there
  is !!a{injective} $f\colon \Domain{M} \to \Domain{M}$ with $\dom{f}
  \neq \Domain{M}$, then $s(u) = f$ is such a variable
  assignment.
\end{proof}

A set $M$ is !!{enumerable} if there is an enumeration
\[
m_0, m_1, m_2, \dots
\]
of its !!{element}s (without repetitions but possibly finite).  Such an enumeration exists
iff there is !!a{element} $z \in M$ and a function $f\colon M \to M$
such that $z$, $f(z)$, $f(f(z))$, \dots, are all the !!{element}s of~$M$. For
if the enumeration exists, $z = m_0$ and $f(m_k) = m_{k+1}$ (or
$f(m_k) = m_k$ if $m_k$ is the last !!{element} of the enumeration)
are the requisite !!{element} and function. On the other hand, if such
a $z$ and $f$ exist, then $z$, $f(z)$, $f(f(z))$, \dots, is an
enumeration of~$M$, and $M$ is !!{enumerable}.  We can express the
existence of $z$ and~$f$ in second-order logic to produce a
!!{sentence} true in !!a{structure} iff the !!{structure} is
!!{enumerable}:
\[
\fn{Count} \ident
\lexists[z][\lexists[u][\lforall[X][((X(z) \land
      \lforall[x][(X(x) \lif X(u(x)))]) \lif \lforall[x][X(x)])]]]
\]

\begin{prop}
$\Sat{M}{\fn{Count}}$ iff $\Domain{M}$ is !!{enumerable}.
\end{prop}

\begin{proof}
Suppose $\Domain{M}$ is !!{enumerable}, and let $m_0$, $m_1$, \dots, be an
enumeration. By removing repetitions we can guarantee that no $m_k$
appears twice. Define $f(m_k) = m_{k+1}$ and let $s(z) = m_0$ and
$s(u) = f$. We show that
\[
\Sat{M}{\lforall[X][((X(z) \land \lforall[x][(X(x) \lif X(u(x)))])
    \lif \lforall[x][X(x)])]}[s]
\]
Suppose $M \subseteq \Domain{M}$ is arbitrary. Suppose further that
$\Sat{M}{(X(z) \land \lforall[x][(X(x) \lif
X(u(x)))])}[\Subst{s}{M}{X}]$. Then $\Subst{s}{M}{X}(z) \in M$ and
whenever $x \in M$, also $(\Subst{s}{M}{X}(u))(x) \in M$. In other
words, since $\varAssign{\Subst{s}{M}{X}}{s}{X}$, $m_0 \in M$ and if
$x \in M$ then $f(x) \in M$, so $m_0 \in M$, $m_1 = f(m_0) \in M$,
$m_2 = f(f(m_0)) \in M$, etc. Thus, $M = \Domain{M}$, and so
$\Sat{M}{\lforall[x][X(x)]}[\Subst{s}{M}{X}]$. Since $M \subseteq
\Domain{M}$ was arbitrary, we are done:
$\Sat{M}{\fn{Count}}$.

Now assume that $\Sat{M}{\fn{Count}}$, i.e., 
\[
\Sat{M}{\lforall[X][((X(z) \land \lforall[x][(X(x) \lif X(u(x)))])
    \lif \lforall[x][X(x)])]}[s]
\]
for some~$s$. Let $m = s(z)$ and $f = s(u)$ and consider $M = \{m,
f(m), f(f(m)), \dots\}$. $M$ so defined is clearly !!{enumerable}.
Then
\[
\Sat{M}{(X(z) \land \lforall[x][(X(x) \lif X(u(x)))])
    \lif \lforall[x][X(x)]}[\Subst{s}{M}{X}]
\]
by assumption. Also, $\Sat{M}{X(z)}[\Subst{s}{M}{X}]$ since $M \ni m =
\Subst{s}{M}{X}(z)$, and also $\Sat{M}{\lforall[x][(X(x) \lif
X(u(x)))]}[\Subst{s}{M}{X}]$ since whenever $x \in M$ also $f(x) \in
M$. So, since both antecedent and conditional are satisfied, the
consequent must also be: $\Sat{M}{\lforall[x][X(x)]}[\Subst{s}{M}{X}]$. But that
means that $M = \Domain{M}$, and so $\Domain{M}$ is !!{enumerable}
since $M$ is, by definition.
\end{proof}

\begin{prob}
The !!{sentence} $\fn{Inf} \land \fn{Count}$ is true in all and only
!!{denumerable} domains.  Adjust the definition of $\fn{Count}$ so
that it becomes a different !!{sentence} that directly expresses that
the domain is !!{denumerable}, and prove that it does.
\end{prob}

\end{document}

