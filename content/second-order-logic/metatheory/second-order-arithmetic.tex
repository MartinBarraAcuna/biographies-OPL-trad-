% Part: second-order-logic
% Chapter: metatheory
% Section: second-order-arithmetic

\documentclass[../../../include/open-logic-section]{subfiles}

\begin{document}

\olfileid{sol}{met}{spa}

\olsection{Second-order Arithmetic}

Recall that the theory $\Th{PA}$ of Peano arithmetic includes the
eight axioms of~$\Th{Q}$,
\begin{align*}
& \lforall[x][\eq/[x'][\Obj 0]]\\
& \lforall[x][\lforall[y][(\eq[x'][y'] \lif \eq[x][y])]]\\
& \lforall[x][(\eq[x][\Obj 0] \lor \lexists[y][\eq[x][y']])]\\ 
& \lforall[x][\eq[(x + \Obj 0)][x]]\\
& \lforall[x][\lforall[y][\eq[(x + y')][(x + y)']]]\\
& \lforall[x][\eq[(x \times \Obj 0)][\Obj 0]]\\
& \lforall[x][\lforall[y][\eq[(x \times y')][((x \times y) + x)]]]\\
& \lforall[x][\lforall[y][(x < y \liff \lexists[z][\eq[(z' + x)][y]])]]\\
\intertext{plus all sentences of the form}
& (!A(\Obj 0) \land \lforall[x][(!A(x) \lif !A(x'))]) \lif \lforall[x][!A(x)].
\end{align*}
The latter is a ``schema,'' i.e., a pattern that generates infinitely
many !!{sentence}s of the language of arithmetic, one for each
!!{formula}~$!A(x)$. We call this schema the (first-order)
\emph{axiom schema of induction}. 
In \emph{second-order} Peano arithmetic~$\Th{PA^2}$, induction
can be stated as a single sentence. $\Th{PA^2}$
consists of the first eight axioms above plus the (second-order)
\emph{induction axiom}:
\[
\lforall[X][(X(\Obj 0) \land \lforall[x][(X(x) \lif X(x'))]) \lif \lforall[x][X(x)]].
\]
It says that if a subset~$X$ of the !!{domain}
contains~$\Assign{\Obj{0}}{M}$ and with any~$x \in \Domain{M}$ also
contains~$\Assign{\prime}{M}(x)$ (i.e., it is ``closed under
successor'') it contains everything in the !!{domain} (i.e., $X =
\Domain{M})$.

The induction axiom guarantees that any !!{structure} satisfying it
contains only those !!{element}s of~$\Domain{M}$ the axioms require to
be there, i.e., the values of~$\num{n}$ for $n \in \Nat$. A model of
$\Th{PA^2}$ contains no non-standard numbers.

\begin{thm}
\ollabel{thm:sol-pa-standard}
If $\Sat{M}{\Th{PA^2}}$ then $\Domain{M} =
\Setabs{\Value{\num{n}}{M}}{n \in \Nat}$.
\end{thm}

\begin{proof}
Let $N = \Setabs{\Value{\num{n}}{M}}{n \in \Nat}$, and suppose
$\Sat{M}{\Th{PA^2}}$. Of course, for any $n \in \Nat$,
$\Value{\num{n}}{M} \in \Domain{M}$, so $N \subseteq \Domain{M}$.

Now for inclusion in the other direction. Consider a variable
assignment $s$ with $s(X) = N$. By assumption,
\begin{align*}
\Sat{M}{\lforall[X][(X(\Obj 0) \land \lforall[x][(X(x)
      \lif X(x'))]) \lif \lforall[x][X(x)]]}, & \text{ thus}\\
\Sat{M}{(X(\Obj 0) \land \lforall[x][(X(x) \lif X(x'))]) \lif
  \lforall[x][X(x)]}[s]. & 
\end{align*}
Consider the antecedent of this conditional.  $\Value{\Obj{0}}{M} \in
N$, and so $\Sat{M}{X(\Obj{0})}[s]$.  The second conjunct,
$\lforall[x][(X(x) \lif X(x'))]$ is also satisfied. For suppose $x \in
N$. By definition of~$N$, $x = \Value{\num{n}}{M}$ for some~$n$.  That
gives $\Assign{\prime}{M}(x) = \Value{\num{n+1}}{M} \in N$. So,
$\Assign{\prime}{M}(x) \in N$.

We have that $\Sat{M}{X(\Obj 0) \land \lforall[x][(X(x) \lif X(x'))]}[s]$.
Consequently, $\Sat{M}{\lforall[x][X(x)]}[s]$. But that means that for
every $x \in \Domain{M}$ we have $x \in s(X) = N$. So, $\Domain{M}
\subseteq N$.
\end{proof}

\begin{cor}
\ollabel{cor:sol-pa-categorical}
Any two models of~$\Th{PA^2}$ are isomorphic.
\end{cor}

\begin{proof}
By \olref{thm:sol-pa-standard}, the domain of any model of $\Th{PA^2}$
is exhausted by $\Value{\num{n}}{M}$. Any such model is also a model
of~$\Th{Q}$. By \olref[mod][mar][stm]{prop:thq-standard}, any such
model is standard, i.e., isomorphic to~$\Struct{N}$.
\end{proof}

Above we defined $\Th{PA^2}$ as the theory that contains the first eight 
arithmetical axioms plus the second-order induction axiom.  In fact, thanks
to the expressive power of second-order logic, only the \emph{first two} 
of the arithmetical axioms plus induction are needed for second-order Peano 
arithmetic.

\begin{prop}\ollabel{prop:sol-pa-definable}
Let $\Th{PA^{2\dagger}}$ be the second-order theory containing the
first two arithmetical axioms (the successor axioms) and the
second-order induction axiom. Then $\le$, $+$, and $\times$ are definable
in $\Th{PA^{2\dagger}}$.
\end{prop}

\begin{proof}
To show that $\le$ is definable, we have to find a formula~$!A_\le(x, y)$
such that $\Sat{N}{!A_\le(\num n, \num m)}$ iff $n \le m$. Consider the formula
\[
!B(x, Y) \ident Y(x) \land \lforall[y][(Y(y) \lif Y(y'))]
\]
Clearly, $!B(\num n, Y)$ is satisfied by a set $Y \subseteq \Nat$ iff
$\Setabs{m}{n \le m} \subseteq Y$, so we can take $!A_\le(x, y) \ident
\lforall[Y][(!B(x, Y) \lif Y(y))]$.

To see that addition is definable observe that $k+l=m$ iff there is a
function $u$ such that $u(0)=k$, $u(n')=u(n)'$ for all $n$, and $m = u(l)$. 
We can use this equivalence to define addition in $\Th{PA^{2\dagger}}$ 
by the following formula:
\[
!A_+(x,y,z) \ident \lexists[u][(u(\Obj 0)=x \land \lforall[w][u(x')=u
 (x)'] \land u(y)=z)]
\] 
It should be clear that $\Sat{N}{!A_+(\num k, \num l, \num m)}$ 
iff $k+l=m$.
\end{proof}

\begin{prob}
Complete the proof of \olref[sol][met][spa]{prop:sol-pa-definable}.
\end{prob}

\end{document}
