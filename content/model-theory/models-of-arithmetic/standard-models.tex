% Part: model-theory
% Chapter: models-of-arithmetic
% Section: standard-models

\documentclass[../../../include/open-logic-section]{subfiles}

\begin{document}

\olfileid{mod}{mar}{stm}
\olsection{Standard Models of Arithmetic}

The language of arithmetic~$\Lang{L_A}$ is obviously intended to be
about numbers, specifically, about natural numbers. So, ``the''
standard model~$\Struct{N}$ is special: it is the model we want to
talk about. But in logic, we are often just interested in structural
properties, and any two !!{structure}s that are isomorphic share
those.  So we can be a bit more liberal, and consider any
!!{structure} that is isomorphic to~$\Struct{N}$ ``standard.''

\begin{defn}
A !!{structure} for $\Lang{L_A}$ is \emph{standard} if it is
isomorphic to~$\Struct{N}$.
\end{defn}

\begin{prop}
\ollabel{prop:standard-domain} If !!a{structure}~$\Struct{M}$ is standard,
then its domain is the set of values of the standard numerals, i.e.,
\[
\Domain{M} = \Setabs{\Value{\num{n}}{M}}{n \in \Nat}
\]
\end{prop}

\begin{proof}
Clearly, every $\Value{\num{n}}{M} \in \Domain{M}$. We just have to
show that every $x \in \Domain{M}$ is equal to $\Value{\num{n}}{M}$
for some~$n$.  Since $\Struct{M}$ is standard, it is isomorphic
to~$\Struct{N}$. Suppose $g\colon \Nat \to \Domain{M}$ is an
isomorphism. Then $g(n) = g(\Value{\num{n}}{N}) =
\Value{\num{n}}{M}$. But for every $x \in \Domain{M}$, there is an~$n
\in \Nat$ such that $g(n) = x$, since $g$ is !!{surjective}.
\end{proof}

\begin{explain}
If a structure~$\Struct{M}$ for $\Lang{L_A}$ is standard, the elements
of its !!{domain} can all be named by the standard numerals $\num{0}$,
$\num{1}$, $\num{2}$, \dots, i.e., the terms $\Obj{0}$, $\Obj{0}'$,
$\Obj{0}''$, etc. Of course, this does not mean that the !!{element}s
of $\Domain{M}$ \emph{are} the numbers, just that we can pick them out
the same way we can pick out the numbers in~$\Domain{N}$.
\end{explain}

\begin{prob}
Show that the converse of \olref[mod][mar][stm]{prop:standard-domain}
is false, i.e., give an example of !!a{structure}~$\Struct{M}$ with
$\Domain{M} = \Setabs{\Value{\num{n}}{M}}{n \in \Nat}$ that is not
isomorphic to~$\Struct{N}$.
\end{prob}

\begin{prop}
\ollabel{prop:thq-standard}
If $\Sat{M}{\Th{Q}}$, and $\Domain{M} = \Setabs{\Value{\num{n}}{M}}{n
  \in \Nat}$, then $\Struct{M}$ is standard.
\end{prop}

\begin{proof}
We have to show that $\Struct{M}$ is isomorphic
to~$\Struct{N}$. Consider the function $g\colon \Nat \to \Domain{M}$
defined by $g(n) = \Value{\num{n}}{M}$. By the hypothesis, $g$ is
!!{surjective}.  It is also !!{injective}: $\Th{Q} \Proves
\eq/[\num{n}][\num{m}]$ whenever $n \neq m$. Thus, since
$\Sat{M}{\Th{Q}}$, $\Sat{M}{\eq/[\num{n}][\num{m}]}$, whenever $n \neq
m$. Thus, if $n \neq m$, then $\Value{\num{n}}{M} \neq
\Value{\num{m}}{M}$, i.e., $g(n) \neq g(m)$.

We also have to verify that $g$ is an isomorphism.
\begin{enumerate}
\item We have $g(\Assign{\Obj{0}}{N}) = g(0)$ since,
  $\Assign{\Obj{0}}{N} = 0$. By definition of~$g$, $g(0) =
  \Value{\num{0}}{M}$. But $\num{0}$ is just $\Obj{0}$, and the value
  of a term which happens to be !!a{constant} is given by what the
  !!{structure} assigns to that !!{constant}, i.e.,
  $\Value{\Obj{0}}{M} = \Assign{\Obj{0}}{M}$. So we have
  $g(\Assign{\Obj{0}}{N}) = \Assign{\Obj{0}}{M}$ as required.
\item $g(\Assign{\prime}{N}(n)) = g(n+1)$, since $\prime$ in
  $\Struct{N}$ is the successor function on~$\Nat$. Then, $g(n+1) =
  \Value{\num{n+1}}{M}$ by definition of~$g$. But $\num{n+1}$ is the
  same term as $\num{n}'$, so $\Value{\num{n+1}}{M} =
  \Value{\num{n}'}{M}$. By the definition of the value function, this
  is $= \Assign{\prime}{M}(\Value{\num{n}}{M})$. Since
    $\Value{\num{n}}{M} = g(n)$ we get $g(\Assign{\prime}{N}(n)) =
  \Assign{\prime}{M}(g(n))$.
\item $g(\Assign{+}{N}(n,m)) = g(n+m)$, since $+$ in $\Struct{N}$ is
  the addition function on~$\Nat$. Then, $g(n+m) =
  \Value{\num{n+m}}{M}$ by definition of~$g$. But $\Th{Q} \Proves
  \num{n+m} = (\num{n} + \num{m})$, so $\Value{\num{n+m}}{M} =
  \Value{\num{n}+\num{m}}{M}$. By the definition of the value
  function, this is $=
  \Assign{+}{M}(\Value{\num{n}}{M},\Value{\num{m}}{M})$. Since
    $\Value{\num{n}}{M} = g(n)$ and $\Value{\num{m}}{M} = g(m)$, we
    get $g(\Assign{+}{N}(n, m)) = \Assign{+}{M}(g(n), g(m))$.
\item $g(\Assign{\times}{N}(n, m)) = \Assign{\times}{M}(g(n), g(m))$:
  Exercise.
\item $\tuple{n,m} \in \Assign{<}{N}$ iff $n < m$. If $n < m$, then
  $\Th{Q} \Proves \num{n} < \num{m}$, and also $\Sat{M}{\num{n} <
  \num{m}}$. Thus $\tuple{\Value{\num{n}}{M}, \Value{\num{m}}{M}} \in
  \Assign{<}{M}$, i.e., $\tuple{g(n), g(m)} \in \Assign{<}{M}$. If $n
  \not< m$, then $\Th{Q} \Proves \lnot \num{n} < \num{m}$, and
  consequently $\Sat/{M}{\num{n} < \num{m}}$. Thus, as before,
  $\tuple{g(n), g(m)} \notin \Assign{<}{M}$. Together, we get:
  $\tuple{n,m} \in \Assign{<}{N}$ iff $\tuple{g(n), g(m)} \in
  \Assign{<}{M}$.
\end{enumerate}
\end{proof}

\begin{explain}
The function~$g$ is the most obvious way of defining a mapping from
$\Nat$ to the domain of any other !!{structure}~$\Struct{M}$ for
$\Lang{L_A}$, since every such $\Struct{M}$ contains !!{element}s named
by $\num{0}$, $\num{1}$, $\num{2}$, etc. So it isn't surprising that
if $\Struct{M}$ makes at least some basic statements about the
$\num{n}$'s true in the same way that~$\Struct{N}$ does, and $g$ is
also bijective, then $g$ will turn into an isomorphism.  In fact, if
$\Domain{M}$ contains no !!{element}s other than what the $\num{n}$'s
name, it's the only one.
\end{explain}

\begin{prop}
\ollabel{prop:thq-unique-iso} If $\Struct{M}$ is standard, then $g$ from
the proof of \olref{prop:thq-standard} is the only isomorphism from $\Struct{N}$ to~$\Struct{M}$.
\end{prop}

\begin{proof}
Suppose $h\colon \Nat \to \Domain{M}$ is an isomorphism between
$\Struct{N}$ and~$\Struct{M}$. We show that $g = h$ by induction
on~$n$. If $n = 0$, then $g(0) = \Assign{\Obj{0}}{M}$ by definition
of~$g$. But since $h$ is an isomorphism, $h(0) =
h(\Assign{\Obj{0}}{N}) =\Assign{\Obj{0}}{M}$, so $g(0) = h(0)$.

Now consider the case for $n+1$. We have
\begin{align*}
  g(n+1) & = \Value{\num{n+1}}{M} \text{ by definition of~$g$}\\
  & = \Value{\num{n}'}{M} \text{ since $\num{n+1}\ident \num{n}'$}\\
  & = \Assign{\prime}{M}(\Value{\num{n}}{M}) 
    \text{ by definition of $\Value{t'}{M}$}\\
  & = \Assign{\prime}{M}(g(n))  \text{ by definition of~$g$}\\
  & = \Assign{\prime}{M}(h(n)) \text{ by induction hypothesis}\\
  & = h(\Assign{\prime}{N}(n)) \text{ since $h$ is an isomorphism}\\
  & = h(n+1)
\end{align*}
\end{proof}

\begin{explain}
For any !!{denumerable} set~$M$, there's !!a{bijection} between
$\Nat$ and $M$, so every such set~$M$ is potentially the !!{domain} of
a standard model~$\Struct{M}$. In fact, once you pick an object $z \in
M$ and a suitable function $s$ as $\Assign{\Obj{0}}{M}$ and
$\Assign{\prime}{M}$, the interpretations of $+$, $\times$, and $<$ is
already fixed.  Only functions~$s\colon M \to M \setminus \{z\}$ that
are both !!{injective} and !!{surjective} are suitable in a standard
model as~$\Assign{\prime}{M}$.  The range of $s$ cannot contain~$z$,
since otherwise $\lforall[x][\eq/[\Obj 0][x']]$ would be false. That
!!{sentence} is true in~$\Struct{N}$, and so $\Struct{M}$ also has to
make it true. The function~$s$ has to be !!{injective}, since the
successor function~$\Assign{\prime}{N}$ in~$\Struct{N}$ is, and that
$\Assign{\prime}{N}$ is !!{injective} is expressed by !!a{sentence}
true in~$\Struct{N}$. It has to be !!{surjective} because otherwise
there would be some $x \in M \setminus \{z\}$ not in the domain
of~$s$, i.e., the !!{sentence} $\lforall[x][(\eq[x][\Obj 0] \lor
\lexists[y][\eq[y'][x]])]$ would be false in~$\Struct{M}$---but it is
true in~$\Struct{N}$.
\end{explain}


\end{document}
