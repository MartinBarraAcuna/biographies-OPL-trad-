% Part: first-order-logic
% Chapter: model-theory
% Section: reducts-and-expansions

\documentclass[../../../include/open-logic-section]{subfiles}

\begin{document}

\olfileid{mod}{bas}{red}
\section{Reductos y Expansiones}

A menudo es útil o necesario comparar lenguajes que tienen
símbolos en común, así como !!{estructuras} para estos lenguajes. El
caso más común es cuando todos los símbolos en !!un{lenguaje}~$\Lang{L}$
también son parte de !!un{lenguaje}~$\Lang{L'}$, es decir, $\Lang{L} \subseteq
\Lang{L'}$. Una $\Lang{L}$-!!{estructura}~$\Struct{M}$ siempre puede
ser expandida a una $\Lang{L'}$-!!{estructura} añadiendo interpretaciones
de los símbolos adicionales, dejando las interpretaciones de los
símbolos comunes iguales. Por otro lado, de una
$\Lang{L'}$-estructura~$\Struct{M'}$ podemos obtener una
$\Lang{L}$-estructura simplemente "olvidando" las interpretaciones de
los símbolos que no ocurren en~$\Lang{L}$.

\begin{defn}
\ollabel{defn:reduct}
Supongamos que $\Lang L \subseteq \Lang L'$, $\Struct M$ es una
$\Lang L$-!!{estructura} y $\Struct M'$ es una $\Lang L'$-!!{estructura}.
$\Struct M$ es el \emph{reducto} de $\Struct M'$ a $\Lang L$, y
$\Struct M'$ es una \emph{expansión} de $\Struct M$ a $\Lang L'$ si y solo si
\begin{enumerate}
\item $\Domain{M} = \Domain{M'}$
\item Para cada !!{constante}~$c \in \Lang L$, $\Assign{c}{M} =
  \Assign{c}{M'}$.
\item Para cada !!{función}~$f \in \Lang L$, $\Assign{f}{M} =
  \Assign{f}{M'}$.
\item Para cada !!{predicado}~$P \in \Lang L$, $\Assign{P}{M} =
  \Assign{P}{M'}$.
\end{enumerate}
\end{defn}

\begin{prop}
\ollabel{prop:reduct}
Si una $\Lang{L}$-!!{estructura}~$\Struct{M}$ es un reducto de una
$\Lang{L'}$-!!{estructura}~$\Struct{M'}$, entonces para todas las
$\Lang{L}$-!!{oraciones}~$!A$,
\[
\Sat{M}{!A} \text{ si y solo si } \Sat{M'}{!A}.
\]
\end{prop}

\begin{proof}
Ejercicio.
\end{proof}

\begin{prob}
Demuestre \olref[mod][bas][red]{prop:reduct}.
\end{prob}

\begin{defn}
Cuando tenemos una $\Lang{L}$-estructura $\Struct{M}$, y $\Lang{L'} =
\Lang{L} \cup \{P\}$ es la expansión de $\Lang{L}$ obtenida añadiendo
un solo !!{predicado}~$P$ de $n$-lugares, y $R \subseteq \Domain{M}^n$
es una relación de $n$-lugares, entonces escribimos $\Expan{M}{R}$ para la
expansión~$\Struct{M'}$ de~$\Struct{M}$ con $\Assign{P}{M'} = R$.
\end{defn}

\end{document}
