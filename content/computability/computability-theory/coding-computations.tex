% Part: computability
% Chapter: computability-theory
% Section: coding-computations

\documentclass[../../../include/open-logic-section]{subfiles}

\begin{document}

\olfileid{cmp}{thy}{cod}
\olsection{Coding Computations}

In every model of computation, it is possible to do the following:
\begin{enumerate}
\item Describe the \emph{definitions} of computable functions in a
  systematic way. For instance, you can think of Turing machine
  specifications, recursive definitions, or programs in a programming
  language as providing these definitions.
\item Describe the complete record of the computation of a function
  given by some definition for a given input. For instance, a Turing
  machine computation can be described by the sequence of
  configurations (state of the machine, contents of the tape) for each
  step of computation.
\item Test whether a putative record of a computation is in fact the
  record of how a computable function with a given definition would be
  computed for a given input.
\item Extract from such a description of the complete record of a
  computation the value of the function for a given input. For
  instance, the contents of the tape in the very last step of a
  halting Turing machine computation is the value.
\end{enumerate}

Using coding, it is possible to assign to each description of a
computable function a numerical \emph{index} in such a way that the
instructions can be recovered from the index in a computable way.
Similarly, the complete record of a computation can be coded by a
single number as well. The resulting arithmetical relation ``$s$
codes the record of computation of the function with index~$e$ for
input~$x$'' and the function ``output of computation sequence with
code~$s$'' are then computable; in fact, they are primitive recursive.

This fundamental fact is very powerful, and allows us to prove a
number of striking and important results about computability,
independently of the model of computation chosen.

\end{document}
