% Part: propositional-logic
% Chapter: syntax-and-semantics
% Section: introduction

\documentclass[../../../include/open-logic-section]{subfiles}

\begin{document}

\olfileid{pl}{syn}{int}

\olsection{Introducción}

La lógica proposicional trata con !!{fórmulas} que se construyen a partir de !!{variables proposicionales} usando los conectivos proposicionales $\lnot$, $\land$, $\lor$, $\lif$ y $\liff$. Intuitivamente, !!una{variable proposicional}~$p$ representa una oración o proposición que es verdadera o falsa. Siempre que se determina el ``valor de verdad'' de la !!{variable proposicional} en !!una{fórmula}, también se determina el valor de verdad de cualquier !!{fórmula} formada a partir de ellas usando conectivos proposicionales. Decimos que la lógica proposicional es \emph{funcional de verdad}, porque su semántica se da por funciones de valores de verdad. En particular, en la lógica proposicional dejamos fuera de consideración cualquier otra determinación de verdad y falsedad, por ejemplo, si algo es necesariamente verdadero en lugar de simplemente contingentemente verdadero, o si se sabe que algo es verdadero, o si algo es verdadero ahora en lugar de ser verdadero o será verdadero. Solo consideramos dos valores de verdad, verdadero ($\True$) y falso ($\False$), y por lo tanto excluimos de la discusión la posibilidad de que una declaración no sea ni verdadera ni falsa, o solo medio verdadera. También nos concentramos solo en conectivos donde el valor de verdad de !!una{fórmula} construida a partir de ellos está completamente determinado por los valores de verdad de sus partes (y no, digamos, por su significado). En particular, si el valor de verdad de los condicionales en inglés es funcional de verdad en este sentido es discutible. El condicional material~$\lif$ lo es; otras lógicas tratan con condicionales que no son funcionales de verdad.

Para desarrollar la teoría y la metateoría de la lógica proposicional funcional de verdad, primero debemos definir la sintaxis y la semántica de sus expresiones. Describiremos una forma de construir !!{fórmulas} a partir de !!{variables proposicionales} usando los conectivos. Son posibles definiciones alternativas. Otros sistemas elegirán símbolos diferentes, seleccionarán conjuntos diferentes de conectivos como primitivos, y usarán paréntesis de manera diferente (o incluso no los usarán en absoluto, como en el caso de la llamada notación polaca). Sin embargo, lo que todos los enfoques tienen en común es que las reglas de formación definen el conjunto de !!{fórmulas} \emph{inductivamente}. Si se hace correctamente, cada expresión puede resultar esencialmente de una sola manera de acuerdo con las reglas de formación. La definición inductiva que resulta en expresiones que son \emph{únicamente legibles} significa que podemos dar significados a estas expresiones usando el mismo método: la definición inductiva.

Dar el significado de las expresiones es el dominio de la semántica. El concepto central en la semántica para la lógica proposicional es el de satisfacción en !!una{valuación}. !!^a{Una valuación}~$\pAssign{v}$ asigna valores de verdad $\True$, $\False$ a las !!{variables proposicionales}. Cualquier !!{valuación} determina un valor de verdad $\pValue{v}(!A)$ para cualquier !!{fórmula}~$!A$. !!^a{Una fórmula} se satisface en !!una{valuación}~$\pAssign{v}$ si y solo si $\pValue{v}(!A) = \True$---escribimos esto como $\pSat{v}{!A}$. Esta relación también se puede definir por inducción sobre la estructura de~$!A$, usando las funciones de verdad para los conectivos lógicos para definir, digamos, la satisfacción de $!A \land !B$ en términos de satisfacción (o no) de $!A$ y~$!B$.

Sobre la base de la relación de satisfacción $\pSat{v}{!A}$ para oraciones, podemos definir las nociones semánticas básicas de tautología, implicación y satisfacibilidad. !!^a{Una fórmula} es una tautología, $\Entails !A$, si cada !!{valuación} la satisface, es decir, $\pValue{v}(!A) = \True$ para cualquier~$\pAssign{v}$. Está implicada por un conjunto de !!{fórmulas}, $\Gamma \Entails !A$, si cada !!{valuación} que satisface todas las !!{fórmulas} en~$\Gamma$ también satisface~$!A$. Y un conjunto de !!{fórmulas} es satisfacible si alguna !!{valuación} satisface todas las !!{fórmulas} en él al mismo tiempo. Debido a que las !!{fórmulas} están definidas inductivamente, y la satisfacción a su vez se define por inducción sobre la estructura de las !!{fórmulas}, podemos usar la inducción para probar propiedades de nuestra semántica y para relacionar las nociones semánticas definidas.


\end{document}
