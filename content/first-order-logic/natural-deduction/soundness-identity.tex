% Part: first-order-logic
% Chapter: natural-deduction
% Section: soundness-identity

\documentclass[../../../include/open-logic-section]{subfiles}

\begin{document}

\olfileid{fol}{ntd}{sid}

\olsection{Soundness with \usetoken{S}{identity}}

\begin{prop}
Natural deduction with rules for $\eq$ is sound.
\end{prop}

\begin{proof}
Any !!{formula} of the form $\eq[t][t]$ is valid, since
for every !!{structure}~$\Struct M$, $\Sat{M}{\eq[t][t]}$. (Note that
we assume the term $t$ to be closed, i.e., it contains no variables,
so variable assignments are irrelevant).

Suppose the last inference in !!a{derivation} is \Elim{\eq}, i.e., the
derivation has the following form:
\begin{prooftree}
  \AxiomC{$\Gamma_1$}
  \RightLabel{$\delta_1$}
  \DeduceC{$\eq[t_1][t_2]$}
  \AxiomC{$\Gamma_2$}
  \RightLabel{$\delta_2$}
  \DeduceC{$!A(t_1)$}
  \RightLabel{\Elim{\eq}}
  \BinaryInfC{$!A(t_2)$}
\end{prooftree}
The premises $\eq[t_1][t_2]$ and $!A(t_1)$ are !!{derive}d from
!!{undischarged} assumptions~$\Gamma_1$ and $\Gamma_2$, respectively.
We want to show that $!A(t_2)$ follows from $\Gamma_1 \cup \Gamma_2$.
Consider !!a{structure}~$\Struct{M}$ with $\Sat{M}{\Gamma_1 \cup
  \Gamma_2}$. By induction hypothesis, $\Sat{M}{!A(t_1)}$ and
$\Sat{M}{\eq[t_1][t_2]}$. Therefore, $\Value{t_1}{M} = \Value{t_2}{M}$. Let
$s$ be any variable assignment, and $m = \Value{t_1}{M} = \Value{t_2}{M}$. By
\olref[fol][syn][ext]{prop:ext-formulas}, $\Sat{M}{!A(t_1)}[s]$ iff
$\Sat{M}{!A(x)}[\Subst{s}{m}{x}]$ iff $\Sat{M}{!A(t_2)}[s]$. Since
$\Sat{M}{!A(t_1)}$, we have $\Sat{M}{!A(t_2)}$.
\end{proof}

\end{document}
