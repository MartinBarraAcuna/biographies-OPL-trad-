% Part: first-order-logic
% Chapter: natural-deduction
% Section: provability-propositional

% verification of properties of provability needed for maximally
% consistent sets in the completeness chapter.

\documentclass[../../../include/open-logic-section]{subfiles}

\begin{document}

\iftag{FOL}
      {\olfileid{fol}{ntd}{ppr}}
      {\olfileid{pl}{ntd}{ppr}}

\olsection{\usetoken{S}{derivability} and the Propositional Connectives}

\begin{explain}
  We establish that the !!{derivability} relation~$\Proves$ of natural
 deduction is strong enough to establish some basic facts
  involving the propositional connectives, such as that $!A \land !B
  \Proves !A$ and $!A, !A \lif !B \Proves !B$ (modus ponens). These
  facts are needed for the proof of the completeness theorem.
\end{explain}

\begin{prop}\ollabel{prop:provability-land}
  \begin{enumerate}
  \item \ollabel{prop:provability-land-left} Both $!A \land !B \Proves
    !A$ and $!A \land !B \Proves !B$
  \item \ollabel{prop:provability-land-right} $!A, !B \Proves !A \land !B$.
  \end{enumerate}
\end{prop}

\begin{proof}
  \begin{enumerate}
  \item We can !!{derive} both
    \begin{prooftree}
      \AxiomC{$!A \land !B$}
      \RightLabel{\Elim{\land}}
      \UnaryInfC{$!A$}
      \DisplayProof\qquad\bottomAlignProof
      \AxiomC{$!A \land !B$}
      \RightLabel{\Elim{\land}}
      \UnaryInfC{$!B$}
    \end{prooftree}
  \item We can !!{derive}:
    \begin{prooftree}
      \AxiomC{$!A$}
      \AxiomC{$!B$}
      \RightLabel{\Intro{\land}}
      \BinaryInfC{$!A \land !B$}
    \end{prooftree}
  \end{enumerate}
\end{proof}


\begin{prop}\ollabel{prop:provability-lor}
  \begin{enumerate}
  \item $!A \lor !B, \lnot !A, \lnot !B$ is inconsistent.
  \item Both $!A \Proves !A \lor !B$ and $!B \Proves !A \lor !B$.
  \end{enumerate}
\end{prop}

\begin{proof}
  \begin{enumerate}
  \item Consider the following !!{derivation}:
    \begin{prooftree}
      \AxiomC{$!A \lor !B$}
      \AxiomC{$\lnot !A$}
      \AxiomC{$\Discharge{!A}{1}$}
      \RightLabel{\Elim{\lnot}}
      \BinaryInfC{$\lfalse$}
      \AxiomC{$\lnot !B$}
      \AxiomC{$\Discharge{!B}{1}$}
      \RightLabel{\Elim{\lnot}}
      \BinaryInfC{$\lfalse$}
      \DischargeRule{\Elim{\lor}}{1}
      \TrinaryInfC{$\lfalse$}
    \end{prooftree}
    This is !!a{derivation} of~$\lfalse$ from !!{undischarged}
    assumptions $!A \lor !B$, $\lnot !A$, and $\lnot !B$.
  \item We can !!{derive} both
    \begin{prooftree}
      \AxiomC{$!A$}
      \RightLabel{\Intro{\lor}}
      \UnaryInfC{$!A \lor !B$}
      \DisplayProof\qquad\bottomAlignProof
      \AxiomC{$!B$}
      \RightLabel{\Intro{\lor}}
      \UnaryInfC{$!A \lor !B$}
    \end{prooftree}
  \end{enumerate}
\end{proof}

\begin{prop}\ollabel{prop:provability-lif}
  \begin{enumerate}
  \item \ollabel{prop:provability-lif-left}  $!A, !A \lif !B \Proves !B$.
  \item \ollabel{prop:provability-lif-right}
    Both $\lnot !A \Proves !A \lif !B$ and $!B \Proves !A \lif !B$.
  \end{enumerate}
\end{prop}

\begin{proof}
  \begin{enumerate}
  \item We can !!{derive}:
    \begin{prooftree}
      \AxiomC{$!A \lif !B$}
      \AxiomC{$!A$}
      \RightLabel{\Elim{\lif}}
      \BinaryInfC{$!B$}
    \end{prooftree}
    
  \item This is shown by the following two !!{derivation}s:
    \begin{prooftree}
      \AxiomC{$\lnot !A$}
      \AxiomC{$\Discharge{!A}{1}$}
      \RightLabel{\Elim{\lnot}}
      \BinaryInfC{$\lfalse$}
      \RightLabel{\FalseInt}
      \UnaryInfC{$!B$}
      \DischargeRule{\Intro{\lif}}{1}
      \UnaryInfC{$!A \lif !B$}
      \DisplayProof\qquad\bottomAlignProof
      \AxiomC{$!B$}
      \RightLabel{\Intro{\lif}}
      \UnaryInfC{$!A \lif !B$}
    \end{prooftree}
    Note that $\Intro{\lif}$ may, but does not have to, !!{discharge} the
    assumption~$!A$.
  \end{enumerate}
\end{proof}

\end{document}
