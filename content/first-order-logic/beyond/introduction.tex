% Part: first-order-logic
% Chapter: beyond
% Section: introduction

\documentclass[../../../include/open-logic-section]{subfiles}

\begin{document}

\olfileid{fol}{byd}{int}

\olsection{Overview}

First-order logic is not the only system of logic of interest: there
are many extensions and variations of first-order logic. A logic
typically consists of the formal specification of a language, usually,
but not always, a deductive system, and usually, but not always, an
intended semantics. But the technical use of the term raises an
obvious question: what do logics that are not first-order logic have
to do with the word ``logic,'' used in the intuitive or philosophical
sense? All of the systems described below are designed to model
reasoning of some form or another; can we say what makes them logical?

No easy answers are forthcoming. The word ``logic'' is used in
different ways and in different contexts, and the notion, like that of
``truth,'' has been analyzed from numerous philosophical stances. For
example, one might take the goal of logical reasoning to be the
determination of which statements are necessarily true, true a priori,
true independent of the interpretation of the nonlogical terms, true
by virtue of their form, or true by linguistic convention; and each of
these conceptions requires a good deal of clarification. Even if one
restricts one's attention to the kind of logic used in mathematics,
there is little agreement as to its scope. For example, in the
\textit{Principia Mathematica}, Russell and Whitehead tried to develop
mathematics on the basis of logic, in the {\em logicist} tradition
begun by Frege. Their system of logic was a form of higher-type logic
similar to the one described below. In the end they were forced to
introduce axioms which, by most standards, do not seem purely logical
(notably, the axiom of infinity, and the axiom of reducibility), but
one might nonetheless hold that some forms of higher-order reasoning
should be accepted as logical. In contrast, Quine, whose ontology does
not admit ``propositions'' as legitimate objects of discourse, argues
that second-order and higher-order logic are really manifestations of
set theory in sheep's clothing; in other words, systems involving
quantification over predicates are not purely logical.

For now, it is best to leave such philosophical issues for a rainy
day, and simply think of the systems below as formal idealizations of
various kinds of reasoning, logical or otherwise.

\end{document}
