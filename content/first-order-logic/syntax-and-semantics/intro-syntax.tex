% Part: first-order-logic
% Chapter: syntax
% Section: intro-syntax

\documentclass[../../../include/open-logic-section]{subfiles}

\begin{document}

\olfileid{fol}{syn}{itx}

\olsection{Introduction}

In order to develop the theory and metatheory of first-order logic, we
must first define the syntax and semantics of its expressions.  The
expressions of first-order logic are terms and !!{formula}s.  Terms
are formed from !!{variable}s, !!{constant}s, and !!{function}s.
!!^{formula}s, in turn, are formed from !!{predicate}s together with
terms (these form the smallest, ``atomic'' !!{formula}s), and then
from atomic !!{formula}s we can form more complex ones using logical
connectives and quantifiers.  There are many different ways to set
down the formation rules; we give just one possible one. Other systems
will chose different symbols, will select different sets of
connectives as primitive, will use parentheses differently (or even not
at all, as in the case of so-called Polish notation).  What all
approaches have in common, though, is that the formation rules define
the set of terms and !!{formula}s \emph{inductively}. If done
properly, every expression can result essentially in only one way
according to the formation rules.  The inductive definition resulting
in expressions that are \emph{uniquely readable} means we can give
meanings to these expressions using the same method---inductive
definition.

\end{document}
