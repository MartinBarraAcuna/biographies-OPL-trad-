% Part: first-order-logic
% Chapter: syntax-and-semantics
% Section: covered-structures

\documentclass[../../../include/open-logic-section]{subfiles}

\begin{document}

\olfileid{fol}{syn}{cov}

\olsection{Covered \printtoken{P}{structure} for First-order Languages}

\begin{explain}
Recall that a term is \emph{closed} if it contains no !!{variable}s.
\end{explain}

\begin{defn}[!!^{value} of closed terms]
If $t$ is a closed term of the language~$\Lang L$ and $\Struct M$ is a
!!{structure} for~$\Lang L$, the \emph{!!{value}}~$\Value{t}{M}$ is
defined as follows:
\begin{enumerate}
\item If $t$ is just the !!{constant}~$c$, then $\Value{c}{M} = \Assign{c}{M}$.
\item If $t$ is of the form $\Atom{f}{t_1, \ldots, t_n}$, then
  \[
  \Value{t}{M} = \Assign{f}{M}(\Value{t_1}{M}, \ldots,
  \Value{t_n}{M}).
  \]
\end{enumerate}
\end{defn}

\begin{defn}[Covered !!{structure}]
A !!{structure} is \emph{covered} if every element of the domain is the
!!{value} of some closed term.
\end{defn}

\begin{ex}
Let ~$\Lang L$ be the language with !!{constant}s $\Obj{zero}$,
$\Obj{one}$, $\Obj{two}$, \dots, the binary !!{predicate}~$<$, and the
binary !!{function}s $+$ and $\times$. Then !!a{structure}~$\Struct
M$ for~$\Lang L$ is the one with domain $\Domain M = \{0, 1, 2, \ldots
\}$ and assignments $\Assign{\Obj{zero}}{M} = 0$,
$\Assign{\Obj{one}}{M} = 1$, $\Assign{\Obj{two}}{M} = 2$, and so
forth. For the binary relation symbol $<$, the set $\Assign{<}{M}$ is
the set of all pairs $\tuple{c_1, c_2} \in \Domain{M}^2$ such that
$c_1$ is less than~$c_2$: for example, $\tuple{1, 3} \in
\Assign{<}{M}$ but $\tuple{2, 2} \notin \Assign{<}{M}$. For the binary
!!{function} $+$, define $\Assign{+}{M}$ in the usual way---for
example, $\Assign{+}{M}(2,3)$ maps to~$5$, and similarly for the
binary !!{function}~$\times$. Hence, the !!{value} of $\Obj{four}$ is
just~$4$, and the !!{value} of $\times(\Obj{two},
+(\Obj{three},\Obj{zero}))$ (or in infix notation, $\Obj{two} \times
(\Obj{three} + \Obj{zero})$) is
\begin{multline*}
\Value{\times(\Obj{two}, +(\Obj{three},\Obj{zero}))}{M} =\\
\begin{aligned}
& =\Assign{\times}{M}(\Value{\Obj{two}}{M}, \Value{+(\Obj{three}, \Obj{zero})}{M})\\
& = \Assign{\times}{M}(\Value{\Obj{two}}{M}, \Assign{+}{M}(\Value{\Obj{three}}{M},
\Value{\Obj{zero}}{M})) \\
& = \Assign{\times}{M}(\Assign{\Obj{two}}{M}, \Assign{+}{M}(\Assign{\Obj{three}}{M},
\Assign{\Obj{zero}}{M})) \\
& = \Assign{\times}{M}(2, \Assign{+}{M}(3, 0)) \\
& = \Assign{\times}{M}(2, 3) \\
& = 6
\end{aligned}
\end{multline*}
\end{ex}

\begin{prob}
Is $\Struct N$, the standard model of arithmetic, covered? Explain.
\end{prob}

\end{document}
