% Part: first-order-logic
% Chapter: introduction
% Section: syntax

\documentclass[../../../include/open-logic-section]{subfiles}

\begin{document}

\olfileid{fol}{int}{syn}

\olsection{Syntax}

We first must make precise what strings of symbols count as
!!{sentence}s of first-order logic.  We'll do this later; for now
we'll just proceed by example.  The basic building blocks---the
vocabulary---of first-order logic divides into two parts. The first
part is the symbols we use to say specific things or to pick out
specific things. We pick out things using !!{constant}s, and we say
stuff about the things we pick out using !!{predicate}s.  E.g, we
might use $\Obj a$ as !!a{constant} to pick out a single thing, and
then say something about it using the !!{sentence}~$\Atom{\Obj P}{\Obj
a}$.  If you have meanings for ``$\Obj a$'' and ``$\Obj P$'' in mind,
you can read $\Atom{\Obj P}{\Obj a}$ as a sentence of English (and you
probably have done so when you first learned formal logic).  Once you
have such simple !!{sentence}s of first-order logic, you can build
more complex ones using the second part of the vocabulary: the logical
symbols (connectives and quantifiers). So, for instance, we can form
expressions like $(\Atom{\Obj P}{\Obj a} \land \Atom{\Obj Q}{\Obj b})$
or~$\lexists[\Obj x][\Atom{\Obj P}{\Obj x}]$.

In order to provide the precise definitions of semantics and the rules
of our !!{derivation} systems required for rigorous meta-logical study, we
first of all have to give a precise definition of what counts as
!!a{sentence} of first-order logic. The basic idea is easy enough to
understand: there are some simple !!{sentence}s we can form from just
!!{predicate}s and !!{constant}s, such as~$\Atom{\Obj P}{\Obj a}$. And
then from these we form more complex ones using the connectives and
quantifiers. But what exactly are the rules by which we are allowed to
form more complex !!{sentence}s?  These must be specified, otherwise
we have not defined ``!!{sentence} of first-order logic'' precisely
enough. There are a few issues. The first one is to get the right
strings to count as !!{sentence}s. The second one is to do this in
such a way that we can give mathematical proofs about \emph{all}
!!{sentence}s. Finally, we'll have to also give precise definitions of
some rudimentary operations with !!{sentence}s, such as ``replace
every $\Obj x$ in~$!A$ by~$\Obj b$.'' The trouble is that the
quantifiers and !!{variable}s we have in first-order logic make it not
entirely obvious how this should be done. E.g., should $\lexists[\Obj
x][\Atom{\Obj P}{\Obj a}]$ count as !!a{sentence}? What about
$\lexists[\Obj x][\lexists[\Obj x][\Atom{\Obj P}{\Obj x}]]$? What
should the result of ``replace $\Obj x$ by~$\Obj b$ in $(\Atom{\Obj
P}{\Obj x} \land \lexists[\Obj x][\Atom{\Obj P}{\Obj x}])$'' be?

\end{document}
