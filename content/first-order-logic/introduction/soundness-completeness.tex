% Part: first-order-logic
% Chapter: introduction
% Section: soundness-completeness

\documentclass[../../../include/open-logic-section]{subfiles}

\begin{document}

\olfileid{fol}{int}{scp}

\olsection{Soundness and Completeness}

We'll also introduce !!{derivation} systems for first-order logic.  There are
many !!{derivation} systems that logicians have developed, but they all define
the same !!{derivability} relation between !!{sentence}s. We say that
$\Gamma$ \emph{!!{derive}s}~$!A$, $\Gamma \Proves !A$, if there is
!!a{derivation} of a certain precisely defined sort. !!^{derivation}s
are always finite arrangements of symbols---perhaps a list of
!!{sentence}s, or some more complicated structure.  The purpose of
!!{derivation} systems is to provide a tool to determine if !!a{sentence} is
entailed by some set~$\Gamma$.  In order to serve that purpose, it
must be true that $\Gamma \Entails !A$ if, and only if, $\Gamma
\Proves !A$.

If $\Gamma \Proves !A$ but not $\Gamma \Entails !A$, our !!{derivation} system
would be too strong, prove too much.  The property that if $\Gamma
\Proves !A$ then $\Gamma \Entails !A$ is called \emph{soundness}, and
it is a minimal requirement on any good !!{derivation} system. On the other
hand, if $\Gamma \Entails !A$ but not $\Gamma \Proves !A$, then our
!!{derivation} system is too weak, it doesn't prove enough.  The property that
if $\Gamma \Entails !A$ then $\Gamma \Proves !A$ is called
\emph{completeness}. Soundness is usually relatively easy to prove (by
induction on the structure of !!{derivation}s, which are inductively
defined). Completeness is harder to prove.

Soundness and completeness have a number of important consequences. If
a set of !!{sentence}s~$\Gamma$ !!{derive}s a contradiction (such as
$!A \land \lnot !A$) it is called \emph{inconsistent}. Inconsistent
$\Gamma$s cannot have any models, they are unsatisfiable. From
completeness the converse follows: any $\Gamma$ that is not
inconsistent---or, as we will say, \emph{consistent}---has a model. In
fact, this is equivalent to completeness, and is the form of
completeness we will actually prove.  It is a deep and perhaps
surprising result: just because you cannot prove $!A \land \lnot !A$
from $\Gamma$ guarantees that there is !!a{structure} that is as
$\Gamma$ describes it.  So completeness gives an answer to the
question: which sets of !!{sentence}s have models? Answer: all and only
consistent sets do.

The soundness and completeness theorems have two important
consequences: the compactness and the L\"owenheim--Skolem theorem.
These are important results in the theory of models, and can be used
to establish many interesting results. We've already mentioned two:
first-order logic cannot express that the !!{domain} of !!a{structure}
is finite or that it is !!{nonenumerable}.

Historically, all of this---how to define syntax and semantics of
first-order logic, how to define good !!{derivation} systems, how to prove that
they are sound and complete, getting clear about what can and cannot
be expressed in first-order languages---took a long time to figure out
and get right.  We now know how to do it, but going through all the
details can still be confusing and tedious.  But it's also important,
because the methods developed here for the formal language of
first-order logic are applied all over the place in logic, computer
science, and linguistics. So working through the details pays off in
the long run.

\end{document}
